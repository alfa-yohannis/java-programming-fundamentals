\documentclass[aspectratio=169, table]{beamer}
\usepackage[utf8]{inputenc}
\usepackage{listings} 

\usetheme{Pradita}

\subtitle{IF120203-Programming Fundamentals}

\title{Session-02:\\\LARGE{Struktur Program, Kelas,\\Object, Atribut, Metode}}
\date[Serial]{\scriptsize {PRU/SPMI/FR-BM-18/0222}}
\author[Pradita]{\small{\textbf{Alfa Yohannis}}}


% Define Java language style for listings
\lstdefinestyle{JavaStyle}{
language=Java,
basicstyle=\ttfamily\footnotesize,
keywordstyle=\color{blue},
commentstyle=\color{gray},
stringstyle=\color{red},
breaklines=true,
showstringspaces=false,
tabsize=2,
captionpos=b,
numbers=left,
numberstyle=\tiny\color{gray},
comment=[l]{//},
morecomment=[s]{/*}{*/},
commentstyle=\color{gray}\ttfamily,
string=[s]{'}{'},
morestring=[s]{"}{"},
%	stringstyle=\color{teal}\ttfamily,
%	showstringspaces=false
}

\lstdefinelanguage{bash} {
	keywords={},
	basicstyle=\ttfamily\small,
	keywordstyle=\color{blue}\bfseries,
	ndkeywords={iex},
	ndkeywordstyle=\color{purple}\bfseries,
	sensitive=true,
	commentstyle=\color{gray},
	stringstyle=\color{red},
	numbers=left,
	numberstyle=\tiny\color{gray},
	breaklines=true,
	frame=lines,
	backgroundcolor=\color{lightgray!10},
	tabsize=2,
	comment=[l]{\#},
	morecomment=[s]{/*}{*/},
	commentstyle=\color{gray}\ttfamily,
	stringstyle=\color{purple}\ttfamily,
	showstringspaces=false
}

\begin{document}

\frame{\titlepage}

% Add table of contents slide
\begin{frame}[fragile]{Contents}
\vspace{15pt}
\begin{columns}[t]
\begin{column}{.5\textwidth}
	\tableofcontents[sections={1-3}]
\end{column}
\begin{column}{.5\textwidth}
	\tableofcontents[sections={4-6}]
\end{column}
\end{columns}
\end{frame}


\section{Struktur Kode Program Java}

\begin{frame}[fragile]{Struktur Kode Program Java}
Kode program Java terdiri dari beberapa komponen utama yang membentuk struktur sebuah aplikasi. Berikut adalah penjelasan mengenai struktur dasar kode program Java:
\end{frame}

\subsection{Kelas (Class)}
\begin{frame}[fragile]{Kelas (Class)}
Semua kode Java harus didefinisikan dalam kelas. Kelas adalah blueprint atau template untuk objek yang akan dibuat. Berikut adalah contoh deklarasi kelas:

\begin{lstlisting}[style=JavaStyle]
public class MyClass {
	// Kode kelas di sini
}
\end{lstlisting}
\end{frame}

\subsection{Metode Utama (Main Method)}
\begin{frame}[fragile]{Metode Utama (Main Method)}
Metode utama adalah titik masuk program Java. Program mulai dieksekusi dari metode ini. Berikut adalah sintaks untuk metode utama:

\begin{lstlisting}[style=JavaStyle]
public static void main(String[] args) {
	// Kode program di sini
}
\end{lstlisting}
\end{frame}

\begin{frame}{Atribut dan Variabel}
	\textbf{Atribut} adalah variabel yang didefinisikan di dalam sebuah kelas dan digunakan untuk merepresentasikan properti atau karakteristik dari sebuah objek. Sementara itu, \textbf{variabel} digunakan untuk menyimpan data dan biasanya dideklarasikan di dalam metode atau blok kode. Perbedaan utama antara atribut dan variabel adalah pada cakupan dan tujuannya:
	\begin{itemize}
		\item \textbf{Atribut} terikat pada keadaan objek dan merupakan bagian dari kelas atau instansinya.
		\item \textbf{Variabel} biasanya digunakan untuk penyimpanan sementara dan perhitungan di dalam metode atau blok kode.
	\end{itemize}
\end{frame}

\subsection{Deklarasi Atribut}
\begin{frame}[fragile]{Deklarasi Atribut}
	Atribut dideklarasikan di dalam kelas untuk merepresentasikan karakteristik dari objek. Berikut adalah contoh deklarasi atribut dalam sebuah kelas:
	
	\begin{lstlisting}[style=JavaStyle]
		class Person {
			String name;
			int age;
		}
	\end{lstlisting}
	Pada contoh ini, \texttt{name} dan \texttt{age} adalah atribut dari kelas \texttt{Person}.
\end{frame}

\subsection{Deklarasi Variabel}
\begin{frame}[fragile]{Deklarasi Variabel}
	Variabel digunakan untuk menyimpan data. Variabel harus dideklarasikan dengan tipe data sebelum digunakan. Berikut adalah contoh deklarasi variabel:
	
	\begin{lstlisting}[style=JavaStyle]
		int age = 30;
		String name = "John";
	\end{lstlisting}
	Pada contoh ini, \texttt{age} dan \texttt{name} adalah variabel yang menyimpan data secara lokal di dalam metode atau blok kode.
\end{frame}

\subsection{Metode (Methods)}
\begin{frame}[fragile]{Metode (Methods)}
Metode adalah blok kode yang melakukan tugas tertentu dan dapat dipanggil dari bagian lain program. Berikut adalah contoh metode:

\begin{lstlisting}[style=JavaStyle]
public void greet() {
	System.out.println("Hello!");
}
\end{lstlisting}
\end{frame}

\subsection{Komentar}
\begin{frame}[fragile]{Komentar}
Komentar digunakan untuk menjelaskan kode dan tidak dieksekusi. Ada dua jenis komentar dalam Java:

\begin{itemize}
\item \texttt{// Ini adalah komentar satu baris}
\item \texttt{/* Ini adalah komentar multi-baris */}
\end{itemize}

\begin{lstlisting}[style=JavaStyle]
// Ini adalah komentar satu baris
/*
Ini adalah komentar multi-baris
*/
\end{lstlisting}
\end{frame}

\subsection{Import}
\begin{frame}[fragile]{Import}
Pernyataan \texttt{import} digunakan untuk memasukkan kelas dari paket lain ke dalam program. Berikut adalah contoh pernyataan import:

\begin{lstlisting}[style=JavaStyle]
import java.util.Scanner;
\end{lstlisting}
\end{frame}

\begin{frame}[fragile]{Contoh Program HelloWorld.java}
\vspace{20pt}
\begin{lstlisting}[style=JavaStyle, caption={Contoh Program HelloWorld.java}]
package hello;
import java.util.Scanner; // Mengimpor kelas Scanner
public class HelloWorld {
	// Metode utama: Titik masuk program
	public static void main(String[] args) {
		// Deklarasi variabel
		String name;
		// Membuat objek Scanner untuk menerima input dari pengguna
		Scanner scanner = new Scanner(System.in);
		// Meminta input dari pengguna
		System.out.print("Masukkan nama Anda: ");
		name = scanner.nextLine();  // Membaca input nama dari pengguna
		// Menampilkan output dengan input pengguna
		System.out.println("Hello " + name + "!");
		// Menutup objek Scanner
		scanner.close();
	}
}
\end{lstlisting}
\end{frame}

\begin{frame}[fragile]{Penjelasan Kode HelloWorld.java - Bagian 1}
\begin{itemize}
\item \texttt{package hello;} - Mendeklarasikan paket tempat kelas ini berada. Paket membantu dalam mengorganisir kode.
\item \texttt{import java.util.Scanner;} - Mengimpor kelas \texttt{Scanner} dari paket \texttt{java.util} untuk menerima input dari pengguna.
\item \texttt{public class HelloWorld \{ \}} - Mendefinisikan kelas publik \texttt{HelloWorld}. Kelas ini berfungsi sebagai blueprint untuk objek.
\item \texttt{public static void main(String[] args) \{ \}} - Metode utama yang dieksekusi pertama kali. Kode program dimulai dari sini.
\item \texttt{String name;} - Deklarasi variabel \texttt{name} yang akan menyimpan input nama pengguna.
\end{itemize}
\end{frame}

\begin{frame}[fragile]{Penjelasan Kode HelloWorld.java - Bagian 2}
\begin{itemize}
\item \texttt{Scanner scanner = new Scanner(System.in);} - Membuat objek \texttt{Scanner} untuk membaca input dari konsol.
\item \texttt{System.out.print("Masukkan nama Anda: ");} - Mencetak pesan ke konsol meminta pengguna untuk memasukkan nama.
\item \texttt{name = scanner.nextLine();} - Membaca input nama dari pengguna dan menyimpannya dalam variabel \texttt{name}.
\item \texttt{System.out.println("Hello " + name + "!");} - Mencetak pesan "Hello [Nama]!" ke konsol dengan nama yang dimasukkan oleh pengguna.
\item \texttt{scanner.close();} - Menutup objek \texttt{Scanner} untuk menghindari kebocoran sumber daya.
\end{itemize}
\end{frame}


\section{Kode Java: Menghitung Panjang Hipotenusa}

\begin{frame}[fragile]{Kode Java: Menghitung Panjang Hipotenusa}
\begin{lstlisting}[style=JavaStyle, caption={Kode Java: MyTest.java}]
package org.pradita.ddp.pertemuan02;
public class MyTest {
	public static void main(String[] args) {
		double a, b;
		a = 3.0;
		b = 4.0;
		double c = Math.sqrt(a * a + b * b);
		System.out.println(c);
	}
}
\end{lstlisting}
\end{frame}

\begin{frame}[fragile]{Penjelasan Kode MyTest.java: Bagian 1}
\vspace{15pt}
\begin{itemize}
\item \texttt{package org.pradita.ddp.pertemuan02;} - Mendeklarasikan paket tempat kelas ini berada. Paket membantu dalam mengorganisir dan mengelompokkan kelas.
\item \texttt{public class MyTest \{ \}} - Mendefinisikan kelas publik \texttt{MyTest}. Kelas ini adalah blueprint dari objek yang akan dibuat.
\item \texttt{public static void main(String[] args) \{ \}} - Metode utama yang dijalankan pertama kali ketika program dieksekusi. Ini adalah titik masuk dari aplikasi Java.
\item \texttt{double a, b;} - Mendeklarasikan dua variabel bertipe \texttt{double} yang akan menyimpan nilai panjang sisi segitiga.
\item \texttt{a = 3.0;} - Menginisialisasi variabel \texttt{a} dengan nilai 3.0.
\item \texttt{b = 4.0;} - Menginisialisasi variabel \texttt{b} dengan nilai 4.0.
\end{itemize}
\end{frame}

\begin{frame}[fragile]{Penjelasan Kode MyTest.java: Bagian 2}
\begin{itemize}
\item \texttt{double c = Math.sqrt(a * a + b * b);} - Menghitung panjang hipotenusa \texttt{c} menggunakan rumus Pythagoras dan fungsi \texttt{Math.sqrt()} untuk menghitung akar kuadrat dari hasil penjumlahan kuadrat \texttt{a} dan \texttt{b}.
\item \texttt{System.out.println(c);} - Mencetak hasil perhitungan panjang hipotenusa ke konsol.
\end{itemize}
Program ini adalah contoh sederhana yang mendemonstrasikan penggunaan operasi matematika dan metode dari kelas \texttt{Math} di Java untuk menyelesaikan masalah geometris. Mengubah nilai dari \texttt{a} dan \texttt{b} memungkinkan perhitungan panjang hipotenusa dari segitiga dengan sisi yang berbeda.
\end{frame}

%%%%%----------------------------------------------


\begin{frame}[fragile]{Kelas Person: Bagian 1}
	\begin{lstlisting}[style=JavaStyle]
		package org.pradita.ddp.pertemuan02;

		public class Person {		
			private String name, lastName;
			private int age;
		\end{lstlisting}
		
		\begin{itemize}
			\item \texttt{package org.pradita.ddp.pertemuan02;} - Mendeklarasikan paket tempat kelas ini berada.
			\item \texttt{public class Person \{ \}} - Mendefinisikan kelas publik \texttt{Person}. Kelas ini adalah blueprint untuk objek yang akan dibuat.
			\item \texttt{private String name, lastName;} - Mendeklarasikan atribut \texttt{name} dan \texttt{lastName} yang menyimpan nama depan dan belakang.
			\item \texttt{private int age;} - Mendeklarasikan atribut \texttt{age} yang menyimpan umur.
		\end{itemize}
	\end{frame}
	
	\begin{frame}[fragile]{Kelas Person: Bagian 2}
		\begin{lstlisting}[style=JavaStyle]
			public Person() {
				this.name = "Charlie";
				this.age = 17;
				this.lastName = "Chaplin";
			}	
			public Person(String name, String lastName, int age) {
				this.name = name;
				this.age = age;
				this.lastName = lastName;
			}
		\end{lstlisting}
		
		\begin{itemize}
			\item \texttt{public Person() \{ \}} - Konstruktor default yang menginisialisasi \texttt{name} dengan "Charlie", \texttt{age} dengan 17, dan \texttt{lastName} dengan "Chaplin".
			\item \texttt{public Person(String name, String lastName, int age) \{ \}} - Konstruktor dengan parameter untuk menginisialisasi \texttt{name}, \texttt{lastName}, dan \texttt{age}.
		\end{itemize}
	\end{frame}
	
	\begin{frame}[fragile]{Kelas Person: Bagian 3}
		\begin{lstlisting}[style=JavaStyle]
			public String getFullName() {
				return name + " " + lastName;
			}	
			public void introduceMyself() {
				System.out.println("My name is " + this.getFullName() + " and my age is " + this.getAge());
			}
		\end{lstlisting}
		
		\begin{itemize}
			\item \texttt{public String getFullName() \{ \}} - Metode yang mengembalikan nama lengkap dengan menggabungkan \texttt{name} dan \texttt{lastName}.
			\item \texttt{public void introduceMyself() \{ \}} - Metode yang mencetak informasi pribadi ke konsol.
		\end{itemize}
	\end{frame}
	
	\begin{frame}[fragile]{Kelas Person: Bagian 4}
	\vspace{10pt}
		\begin{lstlisting}[style=JavaStyle]
			public String getName() {
				return this.name;
			}
			public int getAge() {
				return this.age;
			}	
			public void setName(String name) {
				this.name = name;
			}
			public void setAge(int age) {
				this.age = age;
			} }
	\end{lstlisting}
	\begin{itemize}
		\item \texttt{public String getName() \{ \}} - Metode getter untuk \texttt{name} \textbf{ | } \texttt{public int getAge() \{ \}} - Metode getter untuk \texttt{age}.
		\item \texttt{public void setName(String name) \{ \}} - Metode setter untuk \texttt{name} \textbf{ | }  \texttt{public void setAge(int age) \{ \}} - Metode setter untuk \texttt{age}.
	\end{itemize}
\end{frame}

\begin{frame}[fragile]{Kelas Main: Bagian 1}
\vspace{10pt}
	\begin{lstlisting}[style=JavaStyle]
		package org.pradita.ddp.pertemuan02;
		public class Main {
			public static void main(String[] args) {
				Person person1 = new Person();
				Person person2 = new Person("Alice", "Wonderland", 31);
				System.out.println("Person1's name is " + person1.getName() + ", age " + person1.getAge());
			\end{lstlisting}
			
			\begin{itemize}
				\item \texttt{package org.pradita.ddp.pertemuan02;} - Paket tempat kelas berada.
				\item \texttt{public class Main \{ \}} - Kelas utama program.
				\item \texttt{public static void main(String[] args) \{ \}} - Metode utama program.
				\item \texttt{Person person1 = new Person();} - Objek \texttt{person1} dengan konstruktor default.
				\item \texttt{Person person2 = new Person("Alice", "Wonderland", 31);} - Objek \texttt{person2} dengan parameter.
			\end{itemize}
		\end{frame}

		
\begin{frame}[fragile]{Kelas Main: Bagian 2 (1/2)}
	\begin{lstlisting}[style=JavaStyle]
		System.out.println("Person2's name is " + person2.getName() + ", age " + person2.getAge());
		System.out.println("Person1's fullname is " + person1.getFullName() + ", age " + person1.getAge());
	\end{lstlisting}
	
	\begin{itemize}
		\item \texttt{System.out.println("Person2's name is " + person2.getName() + ", age " + person2.getAge());} - Cetak nama dan umur \texttt{person2}.
		\item \texttt{System.out.println("Person1's fullname is " + person1.getFullName() + ", age " + person1.getAge());} - Cetak nama lengkap dan umur \texttt{person1}.
	\end{itemize}
\end{frame}

\begin{frame}[fragile]{Kelas Main: Bagian 2 (2/2)}
	\begin{lstlisting}[style=JavaStyle]
		System.out.println("Person2's fullname is " + person2.getFullName() + ", age " + person2.getAge());
		person1.introduceMyself();
		person2.setName("Bob");
		person2.introduceMyself();
	\end{lstlisting}
	
	\begin{itemize}
		\item \texttt{System.out.println("Person2's fullname is " + person2.getFullName() + ", age " + person2.getAge());} - Cetak nama lengkap dan umur \texttt{person2}.
		\item \texttt{person1.introduceMyself();} - Panggil metode \texttt{introduceMyself()} dari \texttt{person1}.
		\item \texttt{person2.setName("Bob");} - Ubah nama \texttt{person2} menjadi "Bob".
		\item \texttt{person2.introduceMyself();} - Panggil \texttt{introduceMyself()} setelah mengubah nama.
	\end{itemize}
\end{frame}


\begin{frame}{Penjelasan Kode}
	\begin{itemize}
		\item \textbf{Kelas} - Blueprint untuk membuat objek, mendefinisikan atribut dan metode.
		\item \textbf{Objek} - Instansi dari kelas yang menggunakan atribut dan metode kelas.
		\item \textbf{Atribut} - Variabel dalam kelas untuk menyimpan data, seperti \texttt{name} dan \texttt{age}.
		\item \textbf{Metode} - Fungsi dalam kelas untuk mengelola atribut dan melakukan tindakan, seperti \texttt{getFullName()}.
	\end{itemize}
	Program ini terdiri dari:
	\begin{itemize}
		\item \textbf{Kelas Person} - Mengelola informasi tentang seseorang.
		\item \textbf{Kelas Main} - Titik masuk program dan penggunaan objek \texttt{Person}.
	\end{itemize}
\end{frame}

\begin{frame}[fragile]{Latihan 1}
\vspace{15pt}
\textbf{Tugas:} Tambahkan metode untuk menghitung keliling di kelas \texttt{Rectangle}. Buat objek \texttt{Rectangle} di kelas \texttt{Main} dan cetak hasilnya.
\begin{lstlisting}[style=JavaStyle]
package org.pradita.ddp.pertemuan02;
public class Rectangle {
	private double length, width;
	public Rectangle(double length, double width) { this.length = length; this.width = width; }
	public double calculatePerimeter() { return 2 * (length + width); }
}
package org.pradita.ddp.pertemuan02;
public class Main {
	public static void main(String[] args) {
		Rectangle rectangle = new Rectangle(5.0, 3.0);
		System.out.println("The perimeter of the rectangle is " + rectangle.calculatePerimeter() + " units.");
	} }
\end{lstlisting}
\end{frame}

\begin{frame}[fragile]{Latihan 2: Kelas Student}
\vspace{15pt}
\textbf{Tugas:} Buat kelas \texttt{Student} dengan atribut \texttt{name}, \texttt{grade}, dan \texttt{id}. Tambahkan metode untuk menampilkan informasi student.
\begin{lstlisting}[style=JavaStyle]
package org.pradita.ddp.pertemuan02;
public class Student {
	private String name;
	private int grade;
	private String id;
	public Student(String name, int grade, String id) {
		this.name = name;
		this.grade = grade;
		this.id = id;
	}
	public void displayInfo() {
		System.out.println("ID: " + id + ", Name: " + name + ", Grade: " + grade);
	} }
\end{lstlisting}
\end{frame}

\begin{frame}[fragile]{Latihan 2: Kelas Main}
\vspace{15pt}
\textbf{Tugas:} Modifikasi kelas \texttt{Main} untuk menampilkan informasi student.
\begin{lstlisting}[style=JavaStyle]
package org.pradita.ddp.pertemuan02;

public class Main {
	public static void main(String[] args) {
		Student student = new Student("Bob", 90, "S12345");
		student.displayInfo();
	}
}
\end{lstlisting}
\end{frame}


\begin{frame}[fragile]{Latihan 3}
\vspace{15pt}
\textbf{Tugas:} Tambahkan metode di kelas \texttt{Circle} untuk menghitung luas berdasarkan jari-jari. Buat objek \texttt{Circle} di kelas \texttt{Main} dan cetak luasnya.
\begin{lstlisting}[style=JavaStyle]
package org.pradita.ddp.pertemuan02;
public class Circle {
	private double radius;
	public Circle(double radius) { this.radius = radius; }
	public double calculateArea() { return Math.PI * radius * radius; }
}
package org.pradita.ddp.pertemuan02;
public class Main {
	public static void main(String[] args) {
		Circle circle = new Circle(7.0);
		System.out.println("The area of the circle is " + circle.calculateArea() + " square units.");
	}
}
\end{lstlisting}
\end{frame}

\begin{frame}{Soal}
	\begin{itemize}
		\item \textbf{Soal 1:} Buat kelas \texttt{Rectangle} dengan atribut \texttt{width} dan \texttt{height}. Tambahkan metode untuk menghitung luas dan keliling. Implementasikan kelas \texttt{Main} untuk membuat objek \texttt{Rectangle} dan tampilkan hasilnya.
		\item \textbf{Soal 2:} Buat kelas \texttt{BankAccount} dengan atribut \texttt{accountNumber}, \texttt{balance}, dan \texttt{accountHolder}. Tambahkan metode untuk menyetor dan menarik uang, serta menampilkan informasi akun. Implementasikan kelas \texttt{Main} untuk membuat objek \texttt{BankAccount}, lakukan transaksi, dan tampilkan informasi akun.
	\end{itemize}

\end{frame}


\end{document}

