\documentclass[aspectratio=169, table]{beamer}
\usepackage[utf8]{inputenc}
\usepackage{listings} 

\usetheme{Pradita}

\subtitle{IF120203-Programming Fundamentals}

\title{Session-03:\\\LARGE{Enkapsulasi dan Dokumentasi}}
\date[Serial]{\scriptsize {PRU/SPMI/FR-BM-18/0222}}
\author[Pradita]{\small{\textbf{Alfa Yohannis}}}


% Define Java language style for listings
\lstdefinestyle{JavaStyle}{
language=Java,
basicstyle=\ttfamily\footnotesize,
keywordstyle=\color{blue},
commentstyle=\color{gray},
stringstyle=\color{red},
breaklines=true,
showstringspaces=false,
tabsize=2,
captionpos=b,
numbers=left,
numberstyle=\tiny\color{gray},
frame=lines,
backgroundcolor=\color{lightgray!10},
comment=[l]{//},
morecomment=[s]{/*}{*/},
commentstyle=\color{gray}\ttfamily,
string=[s]{'}{'},
morestring=[s]{"}{"},
%	stringstyle=\color{teal}\ttfamily,
%	showstringspaces=false
}

\lstdefinelanguage{bash} {
keywords={},
basicstyle=\ttfamily\small,
keywordstyle=\color{blue}\bfseries,
ndkeywords={iex},
ndkeywordstyle=\color{purple}\bfseries,
sensitive=true,
commentstyle=\color{gray},
stringstyle=\color{red},
numbers=left,
numberstyle=\tiny\color{gray},
breaklines=true,
frame=lines,
backgroundcolor=\color{lightgray!10},
tabsize=2,
comment=[l]{\#},
morecomment=[s]{/*}{*/},
commentstyle=\color{gray}\ttfamily,
stringstyle=\color{purple}\ttfamily,
showstringspaces=false
}

\begin{document}

\frame{\titlepage}

% Add table of contents slide
\begin{frame}[fragile]{Contents}
\vspace{15pt}
\begin{columns}[t]
	\begin{column}{.5\textwidth}
		\tableofcontents[sections={1-3}]
	\end{column}
	\begin{column}{.5\textwidth}
		\tableofcontents[sections={4-6}]
	\end{column}
\end{columns}
\end{frame}

% Section: Enkapsulasi
\section{Enkapsulasi}

\begin{frame}
\frametitle{Enkapsulasi}
\begin{itemize}
	\item Enkapsulasi adalah salah satu prinsip dasar pemrograman berorientasi objek (OOP) yang bertujuan untuk menyembunyikan detail implementasi internal suatu kelas dan hanya menyediakan akses melalui metode yang telah ditentukan.
	\item Membantu dalam mengatur kompleksitas dengan mengisolasi bagian-bagian dari kode dan membatasi akses langsung ke data sensitif.
	\item Diimplementasikan dengan menggunakan modifikator akses seperti \texttt{private}, \texttt{protected}, dan \texttt{public}.
\end{itemize}
\end{frame}

% Subsection: Modifikator Akses
\subsection{Modifikator Akses}

\begin{frame}
\frametitle{Modifikator Akses dalam Java}
\begin{itemize}
	\item \textbf{Private:} Atribut atau metode yang dideklarasikan dengan \texttt{private} hanya dapat diakses dari dalam kelas itu sendiri. Melindungi data dari modifikasi langsung dari luar kelas.
	\item \textbf{Protected:} Atribut atau metode yang dideklarasikan dengan \texttt{protected} dapat diakses oleh kelas yang berada dalam paket yang sama atau oleh subclass.
	\item \textbf{Public:} Atribut atau metode yang dideklarasikan dengan \texttt{public} dapat diakses dari mana saja, baik dari dalam kelas, kelas lain dalam paket yang sama, maupun kelas yang berada di luar paket.
\end{itemize}
\end{frame}

% Section: Dokumentasi
\section{Dokumentasi}

\begin{frame}
\frametitle{Dokumentasi}
\begin{itemize}
	\item Dokumentasi adalah bagian penting dari pemrograman yang membantu pengembang memahami kode dan cara menggunakannya.
	\item Di Java, dokumentasi biasanya dibuat menggunakan komentar dalam kode.
\end{itemize}
\end{frame}

% Subsection: Jenis-Jenis Komentar
\subsection{Jenis-Jenis Komentar}

\begin{frame}
\frametitle{Jenis-Jenis Komentar dalam Java}
\begin{itemize}
	\item \textbf{Komentar Baris Tunggal:} Dimulai dengan \texttt{//}, digunakan untuk menjelaskan bagian-bagian kecil dari kode.
	\item \textbf{Komentar Blok:} Dimulai dengan \texttt{/*} dan diakhiri dengan \texttt{*/}, digunakan untuk menjelaskan bagian kode yang lebih besar atau memberikan informasi tambahan yang lebih mendetail.
	\item \textbf{Komentar Dokumentasi:} Dimulai dengan \texttt{/**} dan diakhiri dengan \texttt{*/}, digunakan untuk menghasilkan dokumentasi otomatis menggunakan alat seperti Javadoc.
\end{itemize}
\end{frame}

% Section: Contoh Kasus
\section{Contoh Kasus}

\begin{frame}
\frametitle{Contoh Kasus}
Konsep enkapsulasi dan dokumentasi akan dibahas menggunakan kode contoh dari kelas \texttt{BankAccount} dan \texttt{BetterBankAccount}.
\end{frame}

% Subsection: Kode Kelas BankAccount.java
\subsection{Kode Kelas BankAccount.java}

% Slide 1
\begin{frame}[fragile]
\frametitle{BankAccount.java (Bagian 1)}
\textbf{Bagian 1:} Deklarasi kelas \texttt{BankAccount} ditampilkan, termasuk atribut \texttt{accountNumber} dan \texttt{balance} yang dienkapsulasi dengan akses \texttt{private}. Konstruktor kelas digunakan untuk menginisialisasi nomor rekening.
\begin{lstlisting}[style=JavaStyle]
	package com.bank;
	
	public class BankAccount {
		
		private String accountNumber;
		private double balance;
		
		public BankAccount(String accountNumber) {
			this.accountNumber = accountNumber;
		}
	\end{lstlisting}
\end{frame}

% Slide 2
\begin{frame}[fragile]
	\frametitle{BankAccount.java (Bagian 2)}
	\textbf{Bagian 2:} Metode \texttt{getAccountNumber()} dan \texttt{getBalance()} ditampilkan. Metode-metode ini berfungsi untuk mengambil nilai atribut yang dienkapsulasi.
	\begin{lstlisting}[style=JavaStyle]
		public String getAccountNumber() {
			return this.accountNumber;
		}
		
		public double getBalance() {
			return this.balance;
		}
	\end{lstlisting}
\end{frame}

% Slide 3
\begin{frame}[fragile]
	\frametitle{BankAccount.java (Bagian 3)}
	\textbf{Bagian 3:} Metode \texttt{save()} dan \texttt{withdraw()} digunakan untuk menambah atau mengurangi saldo akun. Metode ini mengendalikan akses dan modifikasi langsung terhadap saldo melalui enkapsulasi.
	\begin{lstlisting}[style=JavaStyle]
		public void save(double amount) {
			this.balance = this.balance + amount;
		}
		
		public void withdraw(double amount) {
			this.balance = this.balance - amount;
		}
		
	}
\end{lstlisting}
\end{frame}

% Subsection: Kode Kelas BetterBankAccount.java
\subsection{Kode Kelas BetterBankAccount.java}

% Slide 1
\begin{frame}[fragile]
\frametitle{BetterBankAccount.java (Bagian 1)}
\begin{lstlisting}[style=JavaStyle]
	package com.bank;
	
	/***
	* A class that represents the bank account in the real world.
	* @author Alice
	*/
	public class BetterBankAccount {
		
		private String accountNumber;
		private double balance;
	\end{lstlisting}
	\textbf{Bagian 1:} Deklarasi kelas \texttt{BetterBankAccount} menunjukkan bagaimana kelas ini mewakili rekening bank di dunia nyata. Atribut \texttt{accountNumber} dan \texttt{balance} dienkapsulasi dengan akses \texttt{private}.
\end{frame}

% Slide 2
\begin{frame}[fragile]
	\frametitle{BetterBankAccount.java (Bagian 2)}
	\begin{lstlisting}[style=JavaStyle]
		/***
		* 
		* @param accountNumber the account number.
		*/
		public BetterBankAccount(String accountNumber) {
			this.accountNumber = accountNumber;
		}
	\end{lstlisting}
	\textbf{Bagian 2:} Konstruktor \texttt{BetterBankAccount} digunakan untuk menginisialisasi nomor rekening ketika objek dari kelas dibuat.
\end{frame}

% Slide 3
\begin{frame}[fragile]
	\frametitle{BetterBankAccount.java (Bagian 3)}
	\begin{lstlisting}[style=JavaStyle]
		/***
		* The method returns the number of the bank account in String.
		* @return The account number.
		*/
		public String getAccountNumber() {
			return this.accountNumber;
		}
		/***
		* Get the balance of the bank account.
		* @return The balance of the account.
		*/
		public double getBalance() {
			return this.balance;
		}
	\end{lstlisting}
	\textbf{Bagian 3:} Metode \texttt{getAccountNumber()} dan \texttt{getBalance()} menyediakan akses ke nomor rekening dan saldo akun.
\end{frame}

% Slide 4
\begin{frame}[fragile]
	\frametitle{BetterBankAccount.java (Bagian 4)}
	\begin{lstlisting}[style=JavaStyle]
		/***
		* Add amount to the balance.  
		* @param amount The amount to be saved.
		*/
		public void save(double amount) {
			this.balance = this.balance + amount;
		}
	\end{lstlisting}
	\textbf{Bagian 4:} Metode \texttt{save()} digunakan untuk menambah saldo akun. Javadoc menjelaskan parameter yang digunakan.
\end{frame}

% Slide 5
\begin{frame}[fragile]
	\frametitle{BetterBankAccount.java (Bagian 5)}
	\begin{lstlisting}[style=JavaStyle]
		/***
		* Withdraw amount from the balance.
		* @param amount The amount to be withdrawn.
		*/
		public void withdraw(double amount) {
			if (amount > 0) {
				this.balance = this.balance - amount;
			} else {
				System.out.println("WARNING: Amount should be larger than zero!");
			}
		}
	}
\end{lstlisting}
\textbf{Bagian 5:} Metode \texttt{withdraw()} digunakan untuk menarik saldo, dengan pengecekan agar jumlah yang ditarik lebih besar dari nol.
\end{frame}



\subsection{Kode Kelas Main.java}

\begin{frame}[fragile]
\frametitle{Main.java (Bagian 1)}
\begin{lstlisting}[style=JavaStyle]
	package com.creditservice;
	
	import com.bank.BetterBankAccount;
	
	/***
	* The Main class to launch the program.
	* 
	* @author Bob
	*/
	public class Main {
	\end{lstlisting}
	\textbf{Bagian 1:} Kelas \texttt{Main} merupakan kelas utama yang digunakan untuk meluncurkan program. Dokumentasi Javadoc memberikan informasi mengenai kelas ini dan penulisnya.
\end{frame}

\begin{frame}[fragile]
	\frametitle{Main.java (Bagian 2)}
	\begin{lstlisting}[style=JavaStyle]
		/***
		* The main method to launch the program.
		* 
		* @param args Parameters for the main method.
		*/
		public static void main(String[] args) {
			BetterBankAccount account1 = new BetterBankAccount("ABC123");
			System.out.println("Account number: " + account1.getAccountNumber());
			
			System.out.println("Initial: " + account1.getBalance());
		\end{lstlisting}
		\textbf{Bagian 2:} Pada bagian ini, metode \texttt{main()} didefinisikan. Metode ini meluncurkan program dengan membuat instance dari \texttt{BetterBankAccount} dan menampilkan nomor akun serta saldo awal.
	\end{frame}
	
	\begin{frame}[fragile]
		\frametitle{Main.java (Bagian 3)}
		\begin{lstlisting}[style=JavaStyle]
			account1.save(100.0);
			System.out.println("After Saving: " + account1.getBalance());
			
			account1.withdraw(-0.5);
			System.out.println("After Withdrawal: " + account1.getBalance());
		}
		
	}
\end{lstlisting}
\textbf{Bagian 3:} Bagian ini menunjukkan penggunaan metode \texttt{save()} dan \texttt{withdraw()} pada objek \texttt{account1}. Hasil setelah menyimpan dan menarik saldo ditampilkan, serta hasil dari penarikan yang tidak valid.
\end{frame}

\subsection{Penjelasan Kode}

\subsection{Penjelasan Kode}

\begin{frame}
\frametitle{Penjelasan Kode}
Program ini terdiri dari:
\begin{itemize}
	\item \textbf{Kelas BankAccount:}
	\begin{itemize}
		\item Menunjukkan enkapsulasi dengan atribut \texttt{private}.
		\item Metode \texttt{save()} dan \texttt{withdraw()} mengubah saldo tanpa validasi.
	\end{itemize}
	\item \textbf{Kelas BetterBankAccount:}
	\begin{itemize}
		\item Memperbaiki \texttt{BankAccount} dengan validasi pada \texttt{withdraw()}.
		\item Menambahkan dokumentasi Javadoc untuk menjelaskan fungsi dan parameter.
	\end{itemize}
	\item \textbf{Kelas Main:}
	\begin{itemize}
		\item Titik masuk program yang menggunakan \texttt{BetterBankAccount}.
		\item Menampilkan informasi saldo dan hasil operasi ke konsol.
	\end{itemize}
\end{itemize}
\end{frame}

\subsection{Enkapsulasi pada Kode}

\begin{frame}
\frametitle{Enkapsulasi pada Kode}
Enkapsulasi dicapai dengan:
\begin{itemize}
	\item Atribut \texttt{private} di \texttt{BankAccount} dan \texttt{BetterBankAccount}.
	\item Melindungi data dan menghindari akses langsung.
	\item Memastikan perubahan hanya melalui metode yang dikendalikan.
\end{itemize}
\end{frame}


\subsection{Dokumentasi pada Kode}

\begin{frame}
\frametitle{Dokumentasi pada Kode}
Dokumentasi dalam kode ini menggunakan komentar dokumentasi dengan format \texttt{/*** ... */}.
\begin{itemize}
	\item Menjelaskan tujuan kelas dan metode.
	\item Mencakup parameter dan nilai kembaliannya.
	\item Memberikan panduan tentang cara menggunakan kelas.
	\item Contoh: Komentar pada metode \texttt{withdraw()} di \texttt{BetterBankAccount} memberikan peringatan untuk jumlah yang tidak valid.
\end{itemize}
\end{frame}

\section{Cara Menggenerate Javadoc}

\subsection{Menggenerate Javadoc Menggunakan Eclipse}

\begin{frame}
\frametitle{Menggenerate Javadoc Menggunakan Eclipse}
Eclipse menyediakan fitur untuk menghasilkan Javadoc:
\begin{enumerate}
	\item Buka proyek Java di Eclipse.
	\item Klik kanan pada proyek di \textit{Package Explorer}, pilih \textit{Export}.
	\item Pilih \textit{Javadoc} di bawah kategori \textit{Java} dan klik \textit{Next}.
	\item Pilih proyek atau paket untuk disertakan dalam Javadoc.
	\item Tentukan lokasi penyimpanan Javadoc.
	\item Atur opsi Javadoc seperti visibilitas dan penyertaan \textit{source code}.
	\item Klik \textit{Finish} untuk memulai proses.
\end{enumerate}
\end{frame}

\subsection{Menggenerate Javadoc dari Command Line}

\begin{frame}[fragile]
\frametitle{Menggenerate Javadoc dari Command Line}
Menghasilkan Javadoc dari command line:
\begin{enumerate}
	\item Pastikan berada di direktori proyek Java.
	\item Gunakan perintah berikut:
	\begin{lstlisting}[language=bash]
		javadoc -d doc -sourcepath src -subpackages com.bank
	\end{lstlisting}
	\item Penjelasan perintah:
	\begin{itemize}
		\item \texttt{-d doc}: Direktori output untuk file HTML.
		\item \texttt{-sourcepath src}: Lokasi kode sumber.
		\item \texttt{-subpackages com.bank}: Memproses semua paket di bawah \texttt{com.bank}.
	\end{itemize}
	\item Javadoc akan dihasilkan di direktori \texttt{doc}.
\end{enumerate}
\end{frame}

\begin{frame}
\centering
\Huge Silahkan mengerjakan latihan-latihan di modul ...
\end{frame}


\end{document}

