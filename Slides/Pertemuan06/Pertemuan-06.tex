\documentclass[aspectratio=169, table]{beamer}
\usepackage[utf8]{inputenc}
\usepackage{listings} 

\usetheme{Pradita}

\subtitle{IF120203-Programming Fundamentals}

\title{Session-06:\\\LARGE{Perulangan pada Java\\}
\vspace{10pt}}
\date[Serial]{\scriptsize {PRU/SPMI/FR-BM-18/0222}}
\author[Pradita]{\small{\textbf{Alfa Yohannis}}}


% Define Java language style for listings
\lstdefinestyle{JavaStyle}{
language=Java,
basicstyle=\ttfamily\footnotesize,
keywordstyle=\color{blue},
commentstyle=\color{gray},
stringstyle=\color{red},
breaklines=true,
showstringspaces=false,
tabsize=2,
captionpos=b,
numbers=left,
numberstyle=\tiny\color{gray},
frame=lines,
backgroundcolor=\color{lightgray!10},
comment=[l]{//},
morecomment=[s]{/*}{*/},
commentstyle=\color{gray}\ttfamily,
string=[s]{'}{'},
morestring=[s]{"}{"},
%	stringstyle=\color{teal}\ttfamily,
%	showstringspaces=false
}

\lstdefinelanguage{bash} {
keywords={},
basicstyle=\ttfamily\small,
keywordstyle=\color{blue}\bfseries,
ndkeywords={iex},
ndkeywordstyle=\color{purple}\bfseries,
sensitive=true,
commentstyle=\color{gray},
stringstyle=\color{red},
numbers=left,
numberstyle=\tiny\color{gray},
breaklines=true,
frame=lines,
backgroundcolor=\color{lightgray!10},
tabsize=2,
comment=[l]{\#},
morecomment=[s]{/*}{*/},
commentstyle=\color{gray}\ttfamily,
stringstyle=\color{purple}\ttfamily,
showstringspaces=false
}

\begin{document}

\frame{\titlepage}

% Add table of contents slide
\begin{frame}[fragile]{Contents}
\vspace{15pt}
\begin{columns}[t]
\begin{column}{.5\textwidth}
\tableofcontents[sections={1-8}]
\end{column}
\begin{column}{.5\textwidth}
\tableofcontents[sections={9-20}]
\end{column}
\end{columns}
\end{frame}

	\section{Looping di Java}
	\begin{frame}
		\frametitle{Looping di Java}
		Looping adalah salah satu konsep dasar dalam pemrograman yang memungkinkan eksekusi sebuah blok kode berulang kali, selama kondisi tertentu terpenuhi. Di Java, terdapat beberapa jenis loop yang sering digunakan, yaitu:
	\end{frame}
	
	\subsection{Loop \texttt{for}}
	\begin{frame}[fragile]
		\frametitle{Loop \texttt{for}}
		Loop \texttt{for} digunakan ketika jumlah iterasi sudah diketahui sebelumnya. Loop ini memiliki tiga bagian utama: inisialisasi, kondisi, dan iterasi. Sintaks dasar dari \texttt{for} loop adalah sebagai berikut:
		\begin{lstlisting}[style=JavaStyle]
			for (initialization; condition; iteration) {
				// block of code to be executed
			}
		\end{lstlisting}
	\end{frame}
	
	\subsection{Loop \texttt{while}}
	\begin{frame}[fragile]
		\frametitle{Loop \texttt{while}}
		Loop \texttt{while} digunakan ketika jumlah iterasi tidak diketahui dan bergantung pada kondisi yang diberikan. Loop ini akan terus berjalan selama kondisi yang diberikan bernilai \texttt{true}. Berikut adalah sintaks dasar dari \texttt{while} loop:
		\begin{lstlisting}[style=JavaStyle]
			while (condition) {
				// block of code to be executed
			}
		\end{lstlisting}
	\end{frame}
	
	\subsection{Loop \texttt{do-while}}
	\begin{frame}[fragile]
		\frametitle{Loop \texttt{do-while}}
		Loop \texttt{do-while} mirip dengan loop \texttt{while}, namun perbedaannya adalah loop ini akan mengeksekusi blok kode terlebih dahulu, sebelum memeriksa kondisi. Berikut adalah sintaks dasar dari \texttt{do-while} loop:
		\begin{lstlisting}[style=JavaStyle]
			do {
				// block of code to be executed
			} while (condition);
		\end{lstlisting}
	\end{frame}
	
	\subsection{Loop \texttt{for-each}}
	\begin{frame}[fragile]
		\frametitle{Loop \texttt{for-each}}
		Loop \texttt{for-each} digunakan untuk iterasi melalui elemen-elemen dalam koleksi atau array. Berikut adalah sintaks dasar dari \texttt{for-each} loop:
		\begin{lstlisting}[style=JavaStyle]
			for (type element : collection) {
				// block of code to be executed
			}
		\end{lstlisting}
	\end{frame}
	
	\subsection{Contoh Penggunaan Looping}
	\begin{frame}[fragile]
		\frametitle{Contoh Penggunaan Looping}
		Berikut adalah contoh sederhana yang menggunakan berbagai jenis loop di atas untuk menampilkan tabel perkalian:
		\begin{lstlisting}[style=JavaStyle]
			for (int i = 1; i <= 3; i++) {
				for (int j = 1; j <= 3; j++) {
					System.out.print(i + "x" + j + "=" + (i * j) + " ");
				}
				System.out.println();
			}
		\end{lstlisting}
	\end{frame}
	
	\section{Looping pada Tipe Data Primitif di Java}
	\begin{frame}
		\frametitle{Looping pada Tipe Data Primitif di Java}
		Kelas \texttt{PrimitiveLoop} dalam kode berikut ini menampilkan penggunaan berbagai jenis looping untuk melakukan perhitungan perkalian sederhana. Program ini mengilustrasikan penggunaan loop \texttt{while}, \texttt{do-while}, \texttt{for}, dan \texttt{for-in}.
	\end{frame}
	
	\subsection{Kode Java}
	\begin{frame}[fragile]
		\frametitle{Kode Java}
		\begin{lstlisting}[style=JavaStyle]
			package org.alfa.pertemuan07.looping;
			
			public class PrimitiveLoop {
				public static void main(String[] args) {
					executeWhileLoop(2, 1, 10);
					System.out.println();
					executeDoWhile(3, 1, 10);
					System.out.println();
					executeForLoop(4, 1, 10);
					System.out.println();
					executeForIn(5, new int[] { 1, 2, 3, 4, 100, 6, 7, 8, 9, 10 });
				}
				
				public static void executeWhileLoop(int value, int from, int to) {
					while (from <= to) {
						System.out.println(value + " * " + from + " = " + (value * from));
						from++;
					}
				}
				
				public static void executeDoWhile(int value, int from, int to) {
					do {
						System.out.println(value + " * " + from + " = " + (value * from));
						from++;
					} while (from <= to);
				}
				
				public static void executeForLoop(int value, int from, int to) {
					for (int f = from ; f <= to ; f++) {
						System.out.println(value + " * " + f + " = " + (value * f));
					}
				}
				
				public static void executeForIn(int value, int[] array) {
					for (int element : array) {
						System.out.println(value + " * " + element + " = " + (value * element));
					}
				}
			}
		\end{lstlisting}
	\end{frame}
	
	\subsection{Pembahasan}
	\begin{frame}
		\frametitle{Pembahasan}
		Kelas ini menunjukkan bagaimana looping digunakan untuk menghitung hasil perkalian dari sebuah nilai dengan serangkaian angka. Berikut adalah penjelasan masing-masing metode:
		\begin{itemize}
			\item \textbf{Loop \texttt{while} (Metode \texttt{executeWhileLoop}):} 
			Metode ini menggunakan loop \texttt{while} untuk melakukan iterasi dari nilai awal \texttt{from} hingga nilai akhir \texttt{to}.
			
			\item \textbf{Loop \texttt{do-while} (Metode \texttt{executeDoWhile}):} 
			Metode ini mirip dengan loop \texttt{while}, tetapi dalam loop \texttt{do-while}, blok kode dieksekusi setidaknya sekali.
			
			\item \textbf{Loop \texttt{for} (Metode \texttt{executeForLoop}):} 
			Metode ini menggunakan loop \texttt{for} untuk iterasi dari \texttt{from} hingga \texttt{to}.
			
			\item \textbf{Loop \texttt{for-in} (Metode \texttt{executeForIn}):} 
			Metode ini menggunakan loop \texttt{for-in} untuk iterasi melalui elemen-elemen dalam array.
		\end{itemize}
	\end{frame}
	
	\section{Nested Loop dengan \texttt{while} di Java}
	\begin{frame}
		\frametitle{Nested Loop dengan \texttt{while} di Java}
		Kelas \texttt{Latihan} dalam kode berikut ini menampilkan penggunaan nested loop (loop bersarang) untuk mencetak pola angka secara menurun.
	\end{frame}
	
	\subsection{Kode Java}
	\begin{frame}[fragile]
		\frametitle{Kode Java}
		\begin{lstlisting}[style=JavaStyle]
			package org.alfa.pertemuan07.looping;
			
			public class Latihan {
				public static void main (String[] args) {
					int start = 5;
					int end = 1;
					int i = start;
					while (i >= end) {
						int j = i;
						while (j >= end) {
							System.out.print(j);
							j--;
						}
						System.out.println();
						i--;
					}
				}
			}
		\end{lstlisting}
	\end{frame}
	
	\subsection{Pembahasan}
	\begin{frame}
		\frametitle{Pembahasan}
		Program ini menggunakan nested loop untuk mencetak angka secara menurun dalam bentuk pola piramida terbalik. Berikut adalah penjelasan setiap bagian dari kode:
		\begin{itemize}
			\item \textbf{Deklarasi Variabel:} Program dimulai dengan mendeklarasikan dua variabel, \texttt{start} dan \texttt{end}.
			\item \textbf{Loop \texttt{while} Luar:} Loop \texttt{while} pertama mengontrol jumlah baris yang dicetak.
			\item \textbf{Loop \texttt{while} Dalam:} Loop ini mencetak angka-angka pada setiap baris.
			\item \textbf{Mencetak Baris Baru:} Setelah loop dalam selesai, \texttt{System.out.println()} dipanggil.
			\item \textbf{Dekrementasi \texttt{i}:} Setelah mencetak seluruh angka, nilai \texttt{i} dikurangi satu.
		\end{itemize}
	\end{frame}


	
\end{document}
