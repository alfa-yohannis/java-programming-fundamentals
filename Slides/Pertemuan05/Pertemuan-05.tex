\documentclass[aspectratio=169, table]{beamer}
\usepackage[utf8]{inputenc}
\usepackage{listings} 

\usetheme{Pradita}

\subtitle{IF120203-Programming Fundamentals}

\title{Session-05:\\\LARGE{Pengkodisian pada Java\\}
\vspace{10pt}}
\date[Serial]{\scriptsize {PRU/SPMI/FR-BM-18/0222}}
\author[Pradita]{\small{\textbf{Alfa Yohannis}}}


% Define Java language style for listings
\lstdefinestyle{JavaStyle}{
language=Java,
basicstyle=\ttfamily\footnotesize,
keywordstyle=\color{blue},
commentstyle=\color{gray},
stringstyle=\color{red},
breaklines=true,
showstringspaces=false,
tabsize=2,
captionpos=b,
numbers=left,
numberstyle=\tiny\color{gray},
frame=lines,
backgroundcolor=\color{lightgray!10},
comment=[l]{//},
morecomment=[s]{/*}{*/},
commentstyle=\color{gray}\ttfamily,
string=[s]{'}{'},
morestring=[s]{"}{"},
%	stringstyle=\color{teal}\ttfamily,
%	showstringspaces=false
}

\lstdefinelanguage{bash} {
keywords={},
basicstyle=\ttfamily\small,
keywordstyle=\color{blue}\bfseries,
ndkeywords={iex},
ndkeywordstyle=\color{purple}\bfseries,
sensitive=true,
commentstyle=\color{gray},
stringstyle=\color{red},
numbers=left,
numberstyle=\tiny\color{gray},
breaklines=true,
frame=lines,
backgroundcolor=\color{lightgray!10},
tabsize=2,
comment=[l]{\#},
morecomment=[s]{/*}{*/},
commentstyle=\color{gray}\ttfamily,
stringstyle=\color{purple}\ttfamily,
showstringspaces=false
}

\begin{document}

\frame{\titlepage}

% Add table of contents slide
\begin{frame}[fragile]{Contents}
\vspace{15pt}
\begin{columns}[t]
\begin{column}{.5\textwidth}
\tableofcontents[sections={1-8}]
\end{column}
\begin{column}{.5\textwidth}
\tableofcontents[sections={9-20}]
\end{column}
\end{columns}
\end{frame}

\section{Pengkondisian di Java}

\begin{frame}[fragile]{Pengkondisian di Java}
\begin{itemize}
\item \textbf{Pengkondisian}:
\begin{itemize}
\item Konsep penting dalam pemrograman.
\item Memungkinkan pengambilan keputusan berdasarkan kondisi tertentu.
\end{itemize}

\item \textbf{Pengkondisian di Java}:
\begin{itemize}
\item Menggunakan beberapa struktur dasar:
\begin{itemize}
	\item \texttt{if}
	\item \texttt{if-else}
	\item \texttt{switch-case}
	\item Operator ternary
\end{itemize}
\end{itemize}
\end{itemize}
\end{frame}


\subsection{Struktur If}
\begin{frame}[fragile]{Struktur If}
Struktur \texttt{if} digunakan untuk mengeksekusi blok kode tertentu hanya jika kondisi yang diberikan bernilai \texttt{true}. Bentuk dasarnya adalah:

\begin{lstlisting}[style=JavaStyle]
if (kondisi) {
// kode yang akan dieksekusi jika kondisi benar
}
\end{lstlisting}

Contoh penggunaan:

\begin{lstlisting}[style=JavaStyle]
int nilai = 75;
if (nilai >= 70) {
System.out.println("Lulus");
}
\end{lstlisting}
\end{frame}

\subsection{Struktur If-Else}
\begin{frame}[fragile]{Struktur If-Else}
\vspace{20pt}
Struktur \texttt{if-else} memungkinkan kita untuk menentukan blok kode alternatif yang akan dijalankan jika kondisi tidak terpenuhi. Bentuk dasarnya adalah:

\begin{lstlisting}[style=JavaStyle]
if (kondisi) {
// kode yang akan dieksekusi jika kondisi benar
} else {
// kode yang akan dieksekusi jika kondisi salah
}
\end{lstlisting}

Contoh penggunaan:

\begin{lstlisting}[style=JavaStyle]
int nilai = 65;
if (nilai >= 70) {
System.out.println("Lulus");
} else {
System.out.println("Tidak Lulus");
}
\end{lstlisting}
\end{frame}

\subsection{Struktur Nested If}
\begin{frame}[fragile]{Struktur Nested If}
\vspace{20pt}
Struktur \texttt{nested if} atau \texttt{if} bersarang adalah \texttt{if} di dalam \texttt{if}, yang memungkinkan pemeriksaan kondisi tambahan di dalam kondisi utama.

\begin{lstlisting}[style=JavaStyle]
if (kondisi1) {
if (kondisi2) {
	// kode yang akan dieksekusi jika 
	// kedua kondisi benar
}
}
\end{lstlisting}
\end{frame}

\begin{frame}[fragile]{Contoh Nested If}
\vspace{20pt}
Contoh penggunaan:
\begin{lstlisting}[style=JavaStyle]
int nilai = 85;
if (nilai >= 70) {
if (nilai >= 80) {
	System.out.println("Lulus dengan predikat baik");
} else {
	System.out.println("Lulus");
}
}
\end{lstlisting}
\end{frame}

\subsection{Struktur Switch-Case}
\begin{frame}[fragile]{Struktur Switch-Case}
\vspace{25pt}
Struktur \texttt{switch-case} adalah alternatif untuk \texttt{if-else} yang digunakan ketika ada beberapa kemungkinan nilai untuk satu variabel dan masing-masing nilai tersebut memerlukan tindakan berbeda.

\begin{lstlisting}[style=JavaStyle]
switch (variabel) {
case nilai1:
// kode untuk nilai1
break;
case nilai2:
// kode untuk nilai2
break;
default:
// kode untuk semua nilai yang tidak 
// tercantum
}
\end{lstlisting}
\end{frame}

\begin{frame}[fragile]{Contoh Switch-Case}
\vspace{20pt}
Contoh penggunaan:
\begin{lstlisting}[style=JavaStyle]
char grade = 'B';

switch (grade) {
	case 'A':
		System.out.println("Luar Biasa");
		break;
	case 'B':
		System.out.println("Baik");
		break;
	case 'C':
		System.out.println("Cukup");
		break;
	default:
		System.out.println("Nilai tidak valid");
}
\end{lstlisting}
\end{frame}


\subsection{Operator Ternary}
\begin{frame}[fragile]{Operator Ternary}
Operator ternary adalah cara singkat untuk menulis pernyataan \texttt{if-else} sederhana dalam satu baris kode. Bentuk dasarnya adalah:

\begin{lstlisting}[style=JavaStyle]
variabel = (kondisi) ? nilaiJikaTrue : nilaiJikaFalse;
\end{lstlisting}

Contoh penggunaan:

\begin{lstlisting}[style=JavaStyle]
int nilai = 75;
String hasil = (nilai >= 70) ? "Lulus" : "Tidak Lulus";
System.out.println(hasil);
\end{lstlisting}
\end{frame}

\section{Kode Program Penentuan Nilai dan Kelulusan}

\begin{frame}[fragile]{Deklarasi Package dan Import}
\begin{lstlisting}[style=JavaStyle]
package org.alfa.pertemuan05.decisions;

import java.util.Scanner;
\end{lstlisting}
\begin{itemize}
\item Mendeklarasikan package `org.alfa.pertemuan05.decisions` untuk mengorganisir kode.
\item Mengimpor kelas `Scanner` dari `java.util` untuk membaca input dari pengguna.
\end{itemize}
\end{frame}

\begin{frame}[fragile]{Deklarasi Kelas dan Method Utama}
\begin{lstlisting}[style=JavaStyle]
public class Decision {

public static void main(String[] args) {
	
	/** Decisions, Branching, Conditioning, Flow Control **/
	Scanner scanner = new Scanner(System.in);
\end{lstlisting}
\begin{itemize}
	\item Kelas `Decision` dideklarasikan sebagai kelas publik.
	\item Method `main` berfungsi sebagai titik masuk untuk eksekusi program.
	\item Menginisialisasi objek `Scanner` untuk mengambil input dari pengguna.
\end{itemize}
\end{frame}

\begin{frame}[fragile]{Input Nilai dari Pengguna}
\begin{lstlisting}[style=JavaStyle]
	System.out.print("Enter your score (0.0 - 100.0): ");
	
	double score = scanner.nextDouble();
\end{lstlisting}
\begin{itemize}
	\item Menggunakan `System.out.print()` untuk meminta pengguna memasukkan nilai.
	\item Menyimpan input pengguna dalam variabel `score` menggunakan `scanner.nextDouble()`.
\end{itemize}
\end{frame}

\begin{frame}[fragile]{Pengolahan Nilai dan Penentuan Mark}
\begin{lstlisting}[style=JavaStyle, caption={Pengolahan Nilai dan Penentuan Mark}]
	char mark = 'Z';
	mark = getIfMultipleConditionsMark(score);
	
	System.out.println("Your mark is '" + mark + "'");
\end{lstlisting}
\begin{itemize}
	\item Variabel `mark` diinisialisasi dengan nilai default `'Z'`.
	\item Memanggil metode `getIfMultipleConditionsMark(score)` untuk menentukan nilai mark berdasarkan input.
	\item Mencetak nilai mark ke layar.
\end{itemize}
\end{frame}

\begin{frame}[fragile]{Pemeriksaan Kelulusan}
\begin{lstlisting}[style=JavaStyle]
	if (isShortExamPassed(mark)) {
		System.out.println("You have passed the exam");
	} else {
		System.out.println("You failed the exam");
	}
\end{lstlisting}
\begin{itemize}
	\item Menggunakan struktur kontrol `if-else` untuk memeriksa apakah pengguna lulus ujian.
	\item Memanggil metode `isShortExamPassed(mark)` untuk menentukan kelulusan.
	\item Mencetak pesan kelulusan berdasarkan hasil pemeriksaan.
\end{itemize}
\end{frame}

\begin{frame}[fragile]{Metode Penentuan Mark Menggunakan If}
\vspace{20pt}
\begin{lstlisting}[style=JavaStyle]
	public static char getIfMark(double score) {
		char mark = 'E';
		if (score < 10) { mark = 'E'; return mark; }
		if (score < 20) { mark = 'D'; return mark; }
		if (score < 60) { mark = 'C'; return mark; }
		if (score < 80) { mark = 'B'; return mark; }
		if (score >= 80) { mark = 'A'; }
		return mark;
	}
\end{lstlisting}
\begin{itemize}
	\item Metode `getIfMark` digunakan untuk menentukan nilai mark berdasarkan rentang nilai yang diberikan.
	\item Menggunakan beberapa kondisi `if` untuk menentukan mark yang sesuai.
	\item Mengembalikan karakter mark yang sesuai berdasarkan nilai `score`.
\end{itemize}
\end{frame}


\begin{frame}[fragile]{\Large{Metode Penentuan Mark Menggunakan If-Else}}
\vspace{20pt}
\begin{lstlisting}[style=JavaStyle]
	public static char getIfElseMark(double score) {
		char mark = 'E';
		if (score >= 80) {
			mark = 'A';
		} else if (score >= 60) {
			mark = 'B';
		} else if (score >= 40) {
			mark = 'C';
		} else if (score >= 20) {
			mark = 'D';
		} else {
			mark = 'E';
		}
		return mark;
	}
\end{lstlisting}
\end{frame}

\begin{frame}[fragile]{\Large{Metode Penentuan Mark Menggunakan If-Else (2)}}
\vspace{20pt}
\begin{itemize}
	\item Metode `getIfElseMark` juga menentukan nilai mark, tetapi menggunakan struktur `if-else`.
	\item Memeriksa rentang nilai secara berurutan untuk menetapkan mark yang sesuai.
	\item Mengembalikan karakter mark yang sesuai berdasarkan nilai `score`.
\end{itemize}
\end{frame}

\begin{frame}[fragile]{\Large{Metode Penentuan Mark Menggunakan Nested If}}
\vspace{20pt}
\begin{lstlisting}[style=JavaStyle]
	public static char getNestedIf(double score) {
		char mark = 'E';
		if (score >= 20) {
			mark = 'D';
			if (score >= 40) {
				mark = 'C';
				if (score >= 60) {
					mark = 'B';
					if (score >= 80) {
						mark = 'A';
					}
				}
			}
		}
		return mark;
	}
\end{lstlisting}
\end{frame}

\begin{frame}[fragile]{\Large{Metode Penentuan Mark Menggunakan Nested If (2)}}
\begin{itemize}
	\item Metode `getNestedIf` menggunakan struktur `if` bersarang untuk menentukan nilai mark.
	\item Setiap tingkat pemeriksaan nilai dilakukan berdasarkan hasil pemeriksaan sebelumnya.
	\item Mengembalikan karakter mark berdasarkan nilai `score`.
\end{itemize}
\end{frame}

\begin{frame}[fragile]{\Large{Metode Penentuan Mark Menggunakan Multiple Conditions}}
\begin{lstlisting}[style=JavaStyle]
	public static char getIfMultipleConditionsMark(double score) {
		if (score >= 80) { return 'A'; }
		if (score >= 60 && score < 80) { return 'B'; }
		if (score >= 40 && score < 60) { return 'C'; }
		if (score >= 20 && score < 40) { return 'D'; }
		return 'E';
	}
\end{lstlisting}
\begin{itemize}
	\item Metode `getIfMultipleConditionsMark` menggunakan beberapa kondisi untuk menentukan mark.
	\item Menggunakan kondisi gabungan untuk menetapkan mark yang sesuai.
	\item Mengembalikan karakter mark berdasarkan rentang nilai yang diberikan.
\end{itemize}
\end{frame}

\begin{frame}[fragile]{\Large{Metode Pemeriksaan Kelulusan dengan Switch-Case}}
\vspace{20pt}
\begin{lstlisting}[style=JavaStyle]
	public static boolean isExamPassed(char mark) {
		boolean pass = false;
		switch (mark) {
			case 'A': { pass = true; } break;
			case 'B': ass = true; break;
			case 'C': pass = true; break;
			default: break;
		}
		return pass;
	}
\end{lstlisting}
\begin{itemize}
	\item Metode `isExamPassed` menggunakan `switch-case` untuk memeriksa kelulusan berdasarkan mark.
	\item Menetapkan `pass` ke `true` jika mark adalah 'A', 'B', atau 'C'.
	\item Mengembalikan nilai boolean yang menunjukkan apakah ujian telah lulus.
\end{itemize}
\end{frame}

\begin{frame}[fragile]{Metode Pemeriksaan Kelulusan Singkat}
\begin{lstlisting}[style=JavaStyle]
	public static boolean isShortExamPassed(char mark) {
		boolean pass = false;
		pass = (mark == 'D' || mark == 'E') ? false : true;
		return pass;
	}
\end{lstlisting}
\begin{itemize}
	\item Metode `isShortExamPassed` menggunakan kondisi singkat untuk menentukan kelulusan.
	\item Menggunakan operator ternary untuk menetapkan nilai `pass` berdasarkan mark.
	\item Mengembalikan nilai boolean yang menunjukkan apakah pengguna lulus ujian.
\end{itemize}
\end{frame}

\begin{frame}
\centering
\Huge Silahkan mengerjakan latihan-latihan di modul ...
\end{frame}

\end{document}

