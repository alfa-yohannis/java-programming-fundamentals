\chapter{Array dan ArrayList}

\section{Array dan ArrayList di Java}

\subsection{Array}

Array adalah struktur data yang menyimpan sekumpulan elemen dengan tipe yang sama dalam urutan tertentu. Array memiliki ukuran tetap yang ditentukan saat deklarasi. Beberapa karakteristik array di Java adalah:

\begin{itemize}
	\item \textbf{Deklarasi dan Inisialisasi:} Array dapat dideklarasikan dengan menyebutkan tipe data diikuti oleh nama array dan ukuran array. Contoh:
	\begin{lstlisting}[style=JavaStyle]
		int[] numbers = new int[5];
	\end{lstlisting}
	Array dapat diinisialisasi dengan nilai-nilai pada saat deklarasi:
	\begin{lstlisting}[style=JavaStyle]
		int[] numbers = {1, 2, 3, 4, 5};
	\end{lstlisting}
	\item \textbf{Akses Elemen:} Elemen array diakses menggunakan indeks, yang dimulai dari 0:
	\begin{lstlisting}[style=JavaStyle]
		int firstNumber = numbers[0]; // Mengakses elemen pertama
	\end{lstlisting}
	\item \textbf{Ukuran Array:} Ukuran array tetap dan dapat diakses dengan menggunakan \texttt{length}:
	\begin{lstlisting}[style=JavaStyle]
		int size = numbers.length;
	\end{lstlisting}
\end{itemize}

\subsection{ArrayList}

`ArrayList` adalah implementasi dari interface `List` yang menggunakan array dinamis untuk menyimpan elemen. Beberapa karakteristik `ArrayList` di Java adalah:

\begin{itemize}
	\item \textbf{Deklarasi dan Inisialisasi:} `ArrayList` dapat dideklarasikan dan diinisialisasi dengan tipe data generik. Contoh:
	\begin{lstlisting}[style=JavaStyle]
		ArrayList<Integer> list = new ArrayList<>();
	\end{lstlisting}
	\item \textbf{Menambahkan dan Mengakses Elemen:} Elemen dapat ditambahkan dengan metode \texttt{add} dan diakses dengan metode \texttt{get}:
	\begin{lstlisting}[style=JavaStyle]
		list.add(10);
		int element = list.get(0);
	\end{lstlisting}
	\item \textbf{Menghapus Elemen:} Elemen dapat dihapus dengan metode \texttt{remove}:
	\begin{lstlisting}[style=JavaStyle]
		list.remove(0); // Menghapus elemen pada indeks 0
	\end{lstlisting}
	\item \textbf{Ukuran ArrayList:} Ukuran `ArrayList` dapat diakses dengan menggunakan metode \texttt{size}:
	\begin{lstlisting}[style=JavaStyle]
		int size = list.size();
	\end{lstlisting}
\end{itemize}

\subsection{Konversi Antara Array dan ArrayList}

Mengonversi antara `Array` dan `ArrayList` memungkinkan fleksibilitas dalam manipulasi data. Berikut adalah metode umum untuk konversi:

\begin{itemize}
	\item \textbf{ArrayList ke Array:} Gunakan metode \texttt{toArray(T[] a)} atau stream untuk mengonversi `ArrayList` menjadi array:
	\begin{lstlisting}[style=JavaStyle]
		Integer[] array = list.toArray(new Integer[0]);
		Integer[] array2 = list.stream().toArray(Integer[]::new);
	\end{lstlisting}
	\item \textbf{Array ke ArrayList:} Gunakan metode \texttt{Arrays.asList(T... a)} untuk mengonversi array menjadi `ArrayList`:
	\begin{lstlisting}[style=JavaStyle]
		ArrayList<Integer> newList = new ArrayList<>(Arrays.asList(array));
	\end{lstlisting}
\end{itemize}

\section{Penggunaan Array di Java}

\begin{lstlisting}[style=JavaStyle]
	package org.alfa.pertemuan06.arrayandlist;
	
	import java.util.Arrays;
	
	public class ArrayDemo {
		public static void main(String[] args) {
			int[] a = null;
			int[] b = new int[] {};
			int[] c = { 1, 2, 3 };
			double[] d = new double[] { 1, 2, 3, 4 };
			double[] e = Arrays.copyOfRange(d, 1, 4);
			
			System.out.println(Arrays.toString(a));
			System.out.println(Arrays.toString(b));
			System.out.println(Arrays.toString(c));
			System.out.println(Arrays.toString(d));
			System.out.println(Arrays.toString(e));
			
			System.out.print(d[0]);
			System.out.print(", " + d[1]);
			System.out.print(", " + d[3]);
			
			int[] f = new int[3];
			f[2] = 99;
			f[1] = 100;
			f[0] = 31;
			
			System.out.print(f[0]);
			System.out.print(", " + f[1]);
			System.out.print(", " + f[2]);
		}
	}
\end{lstlisting}

\subsection{Pembahasan}

\subsubsection{Deklarasi dan Inisialisasi Array}

Dalam kode di atas, beberapa array dideklarasikan dan diinisialisasi dengan cara yang berbeda:

\begin{itemize}
	\item \textbf{Array \texttt{a}:} Dideklarasikan tetapi tidak diinisialisasi, sehingga nilainya \texttt{null}.
	\item \textbf{Array \texttt{b}:} Dideklarasikan dan diinisialisasi sebagai array kosong. \texttt{b.length} akan mengembalikan \texttt{0}.
	\item \textbf{Array \texttt{c}:} Dideklarasikan dan diinisialisasi dengan tiga elemen \{1, 2, 3\}.
	\item \textbf{Array \texttt{d}:} Dideklarasikan dan diinisialisasi dengan empat elemen \{1, 2, 3, 4\}.
	\item \textbf{Array \texttt{e}:} Dibuat dengan menggunakan metode \texttt{Arrays.copyOfRange()} untuk menyalin elemen dari indeks 1 hingga 3 dari array \texttt{d}. Ini akan menghasilkan array \{2, 3, 4\}.
\end{itemize}

\subsubsection{Pencetakan Array}

Kode diatas menunjukkan beberapa cara untuk mencetak elemen array:
\begin{itemize}
	\item \texttt{System.out.println(Arrays.toString(array))}: Digunakan untuk mencetak elemen array dalam format string yang terpisah oleh koma.
	\item \texttt{System.out.print()}: Digunakan untuk mencetak elemen array satu per satu dengan pemisah yang ditentukan.
\end{itemize}

\subsubsection{Inisialisasi dan Pencetakan Array Baru}

\begin{itemize}
	\item \textbf{Array \texttt{f}:} Dideklarasikan dengan ukuran 3 dan diinisialisasi dengan nilai \{31, 100, 99\}.
	\item \texttt{System.out.print(f[0]); System.out.print(", " + f[1]); System.out.print(", " + f[2]);}: Digunakan untuk mencetak elemen-elemen array \texttt{f} dengan pemisah koma.
\end{itemize}

\section{Penggunaan ArrayList di Java}

\begin{lstlisting}[style=JavaStyle]
	package org.alfa.pertemuan06.arrayandlist;
	
	import java.util.ArrayList;
	import java.util.Arrays;
	
	public class ArrayListDemo {
		public static void main(String[] args) {
			ArrayList<Integer> a = null;
			ArrayList<Integer> b = new ArrayList<Integer>();
			ArrayList<Integer> c = 
			new ArrayList<>(Arrays.asList(73, 101, 6));
			
			b.add(99);
			System.out.println(b);
			b.add(97);
			b.add(0, 98);
			System.out.println(b);
			b.add(97);
			System.out.println(b);
			
			System.out.println("b's Size = " + b.size());
			
			b.remove(0);
			System.out.println(b);
			
			b.remove(97);
			Integer z = 97;
			b.remove(z);
			System.out.println(b);
			
			System.out.println("c(0) = " + c.get(0));
			System.out.println("c(2) = " + c.get(c.size() - 1));
			
			System.out.println(c);
			c.add(1, 999);
			System.out.println("c(0) = " + c.get(0));
			System.out.println("c(1) = " + c.get(1));
			System.out.println(c);
		}
	}
\end{lstlisting}

\subsection{Pembahasan}

\subsubsection{Deklarasi dan Inisialisasi ArrayList}

Dalam kode di atas, beberapa `ArrayList` dideklarasikan dan diinisialisasi dengan cara yang berbeda:

\begin{itemize}
	\item \textbf{ArrayList \texttt{a}:} Dideklarasikan tetapi tidak diinisialisasi, sehingga nilainya \texttt{null}.
	\item \textbf{ArrayList \texttt{b}:} Dideklarasikan dan diinisialisasi sebagai `ArrayList` kosong.
	\item \textbf{ArrayList \texttt{c}:} Dideklarasikan dan diinisialisasi dengan tiga elemen \{73, 101, 6\} menggunakan metode \texttt{Arrays.asList()}.
\end{itemize}

\subsubsection{Operasi pada ArrayList}

Kode diatas menunjukkan beberapa operasi dasar dengan `ArrayList`:

\begin{itemize}
	\item \texttt{add(element)}: Menambahkan elemen ke dalam `ArrayList`. Misalnya, \texttt{b.add(99)} menambahkan nilai 99 ke dalam `b`.
	\item \texttt{add(index, element)}: Menambahkan elemen pada posisi tertentu. Misalnya, \texttt{b.add(0, 98)} menambahkan nilai 98 di posisi indeks 0.
	\item \texttt{remove(index)}: Menghapus elemen pada posisi tertentu. Misalnya, \texttt{b.remove(0)} menghapus elemen pada indeks 0.
	\item \texttt{remove(element)}: Menghapus elemen yang spesifik. Misalnya, \texttt{b.remove(z)} menghapus nilai yang sama dengan nilai variabel \texttt{z}.
	\item \texttt{size()}: Mengembalikan jumlah elemen dalam `ArrayList`. Misalnya, \texttt{b.size()} mengembalikan ukuran dari `b`.
\end{itemize}

\subsubsection{Mengakses dan Memodifikasi Elemen}

\begin{itemize}
	\item \textbf{Mengakses Elemen:} Elemen `ArrayList` dapat diakses menggunakan metode \texttt{get(index)}. Misalnya, \texttt{c.get(0)} mengakses elemen pertama dari `c`.
	\item \textbf{Modifikasi Elemen:} Elemen `ArrayList` dapat dimodifikasi menggunakan metode \texttt{add(index, element)}. Misalnya, \texttt{c.add(1, 999)} menambahkan nilai 999 di posisi indeks 1.
\end{itemize}

\subsubsection{Array of Objects}

Bagian kode yang dikomentari juga menunjukkan penggunaan array objek. `Object[]` dapat digunakan untuk menyimpan berbagai tipe objek:

\begin{itemize}
	\item \textbf{Array \texttt{person}:} Dideklarasikan dengan ukuran 10 dan diisi dengan nilai yang berbeda, termasuk `String` dan `Double`.
	\item \textbf{Array \texttt{people}:} Dideklarasikan dengan ukuran 100 dan diisi dengan array \texttt{person}.
\end{itemize}

\section{Konversi Antara Array dan ArrayList di Java}

\begin{lstlisting}[style=JavaStyle]
	package org.alfa.pertemuan06.arrayandlist;
	
	import java.util.ArrayList;
	import java.util.Arrays;
	
	public class Coversion {
		public static void main(String[] args) {
			ArrayList<Integer> list = new ArrayList<>();
			list.add(3);
			list.add(2);
			list.add(1);
			System.out.println("List: " + list);
			
			Integer[] array = list.toArray(new Integer[0]);
			System.out.println("Array 1: " + Arrays.toString(array));
			
			Integer[] array2 = list.stream().toArray(Integer[]::new);
			System.out.println("Array 2: " + Arrays.toString(array2));
			
			ArrayList<Integer> newList = new ArrayList<>(Arrays.asList(array));
			System.out.println("New List: " + newList);
			
			int[] a = new int[list.size()];
			a[2] = list.get(0); 
			a[1] = list.get(1); 
			a[0] = list.get(2);
			System.out.println("Array a: " + Arrays.toString(a));
			
			ArrayList<Integer> l = new ArrayList<>();
			l.add(a[2]);
			l.add(a[1]);
			l.add(a[0]);
			System.out.println("List l: " + l.toString());
		}
	}
\end{lstlisting}

\subsection{Pembahasan}

\subsubsection{Konversi dari ArrayList ke Array}

Beberapa metode untuk mengonversi `ArrayList` ke array ditunjukkan dalam kode:

\begin{itemize}
	\item \textbf{Metode \texttt{toArray(T[] a)}:} Mengonversi `ArrayList` menjadi array dengan jenis yang sama. Misalnya, \texttt{list.toArray(new Integer[0])} mengonversi `list` menjadi array \texttt{Integer}.
	\item \textbf{Metode Stream \texttt{toArray()} :} Menggunakan stream untuk mengonversi `ArrayList` menjadi array. Misalnya, \texttt{list.stream().toArray(Integer[]::new)} mengonversi `list` menjadi array \texttt{Integer[]}.
\end{itemize}

\subsubsection{Konversi dari Array ke ArrayList}

\begin{itemize}
	\item \textbf{Metode \texttt{Arrays.asList(T... a)}:} Mengonversi array menjadi `ArrayList`. Misalnya, \texttt{new ArrayList<>(Arrays.asList(array))} mengonversi array \texttt{array} menjadi `ArrayList`.
\end{itemize}

\subsubsection{Manipulasi Array dan ArrayList}

\begin{itemize}
	\item \textbf{Array dari List:} Membuat array \texttt{int} dan menyalin elemen dari `ArrayList` ke array tersebut. Misalnya, \texttt{a[2] = list.get(0)} menyalin elemen pertama dari `list` ke posisi ketiga dari array \texttt{a}.
	\item \textbf{List dari Array:} Membuat `ArrayList` baru dari array \texttt{int}. Misalnya, \texttt{l.add(a[2])} menambahkan elemen array ke dalam `ArrayList`.
\end{itemize}

\section{Latihan dan Contoh Kode}

Berikut adalah beberapa latihan untuk membantu Anda memahami penggunaan Array, ArrayList, dan konversi di Java.

\begin{enumerate}
	
	\item \textbf{Latihan 1: Mengelola Data Mahasiswa}
	
	Tugas: Buatlah sebuah program yang mengelola data mahasiswa dengan menggunakan \texttt{ArrayList}. Tambahkan nama mahasiswa ke dalam list, cetak semua nama, hapus nama tertentu, dan cetak ukuran list setelah penghapusan.
	
	\begin{lstlisting}[style=JavaStyle]
		package org.alfa.pertemuan06.arrayandlist;
		
		import java.util.ArrayList;
		
		public class StudentListDemo {
			
			public static void main(String[] args) {
				ArrayList<String> students = new ArrayList<>();
				students.add("Alice");
				students.add("Bob");
				students.add("Charlie");
				
				System.out.println("Students: " + students);
				
				students.remove("Bob");
				System.out.println("Students after removal: " + students);
				System.out.println("Number of students: " + students.size());
			}
		}
	\end{lstlisting}
	
	\item \textbf{Latihan 2: Konversi Array ke ArrayList dan Sebaliknya}
	
	Tugas: Buatlah program yang mengonversi array string menjadi \texttt{ArrayList}, tambahkan beberapa elemen ke \texttt{ArrayList}, konversi kembali ke array, dan cetak hasilnya.
	
	\begin{lstlisting}[style=JavaStyle]
		package org.alfa.pertemuan06.arrayandlist;
		
		import java.util.ArrayList;
		import java.util.Arrays;
		
		public class ConversionDemo {
			
			public static void main(String[] args) {
				String[] array = {"Red", "Green", "Blue"};
				ArrayList<String> list = new ArrayList<>(Arrays.asList(array));
				
				list.add("Yellow");
				list.add("Purple");
				
				String[] newArray = list.toArray(new String[0]);
				System.out.println("New Array: " + Arrays.toString(newArray));
			}
		}
	\end{lstlisting}
	
	\item \textbf{Latihan 3: Array Multi-Dimensi dan ArrayList}
	
	Tugas: Buatlah program yang menggunakan array dua dimensi untuk menyimpan nilai integer. Konversikan baris pertama dari array dua dimensi ke dalam \texttt{ArrayList}, kemudian cetak hasilnya.
	
	\begin{lstlisting}[style=JavaStyle]
		package org.alfa.pertemuan06.arrayandlist;
		
		import java.util.ArrayList;
		import java.util.Arrays;
		
		public class MultiDimArrayDemo {
			
			public static void main(String[] args) {
				int[][] matrix = {
					{1, 2, 3},
					{4, 5, 6},
					{7, 8, 9}
				};
				
				ArrayList<Integer> firstRow = new ArrayList<>();
				for (int num : matrix[0]) {
					firstRow.add(num);
				}
				
				System.out.println("First Row as ArrayList: " + firstRow); // First Row as ArrayList: [1, 2, 3]
			}
		}
	\end{lstlisting}
\end{enumerate}

\section{Soal}

Berikut adalah beberapa latihan yang berkaitan dengan penggunaan Array dan ArrayList di Java:

\begin{enumerate}
	
	\item \textbf{Soal 1:} Tulis program yang mengelola daftar tugas menggunakan \texttt{ArrayList}. Tambahkan beberapa tugas ke dalam daftar, tandai tugas sebagai selesai dengan menghapusnya dari daftar, dan cetak daftar tugas yang tersisa setelah beberapa tugas dihapus.
	
	\item \textbf{Soal 2:} Tulis program yang menginisialisasi array dua dimensi dengan nilai acak antara 1 hingga 100. Konversikan setiap baris dari array dua dimensi ke dalam \texttt{ArrayList}, lalu cetak \texttt{ArrayList} dari setiap baris.
	
	\item \textbf{Soal 3:} Tulis program yang mengelola daftar nilai siswa menggunakan \texttt{ArrayList} dengan tipe data \texttt{Double}. Tambahkan beberapa nilai ke dalam daftar, hitung rata-rata nilai, dan cetak daftar nilai yang lebih besar dari rata-rata tersebut.
	
\end{enumerate}
