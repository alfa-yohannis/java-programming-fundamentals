\chapter{Abstrak dan Pewarisan (Inheritance)}

\section{Kelas Abstrak dan Pewarisan (Inheritance) di Java}

\subsection{Kelas Abstrak}

Kelas abstrak adalah kelas yang tidak dapat diinstansiasi dan sering digunakan sebagai dasar untuk kelas lain. Kelas ini dapat memiliki metode abstrak (metode tanpa implementasi) yang harus diimplementasikan oleh kelas turunannya. Kelas abstrak juga dapat memiliki metode konkret (metode dengan implementasi).

\textbf{Contoh Kelas Abstrak:}

\begin{lstlisting}[style=JavaStyle]
	package edu.example;
	
	public abstract class Shape {
		private String color;
		
		public Shape(String color) {
			this.color = color;
		}
		
		public String getColor() {
			return color;
		}
		
		public abstract double getArea();
	}
\end{lstlisting}

Pada contoh di atas, kelas \texttt{Shape} adalah kelas abstrak yang memiliki metode abstrak \texttt{getArea()} yang harus diimplementasikan oleh kelas turunannya. Kelas ini juga memiliki metode konkret \texttt{getColor()}.

\subsection{Pewarisan (Inheritance)}

Pewarisan memungkinkan sebuah kelas (kelas turunan) untuk mewarisi atribut dan metode dari kelas lain (kelas dasar). Ini memungkinkan penggunaan kembali kode dan mendukung hierarki kelas. Kelas turunan dapat mengakses metode dan atribut dari kelas dasar, serta menambahkan atau memodifikasi fungsionalitas sesuai kebutuhan.

\textbf{Contoh Pewarisan:}

\begin{lstlisting}[style=JavaStyle]
	package edu.example;
	
	public class Rectangle extends Shape {
		private double width;
		private double height;
		
		public Rectangle(String color, double width, double height) {
			super(color);
			this.width = width;
			this.height = height;
		}
		
		@Override
		public double getArea() {
			return width * height;
		}
		
		public double getWidth() {
			return width;
		}
		
		public double getHeight() {
			return height;
		}
	}
\end{lstlisting}

\subsection{Override}

\texttt{Override} adalah konsep dalam pewarisan di mana sebuah metode di kelas turunan menggantikan atau mengubah implementasi dari metode yang diwarisi dari kelas dasar. Kata kunci \texttt{@Override} digunakan untuk menunjukkan bahwa metode tersebut dimaksudkan untuk menggantikan metode yang ada di kelas dasar. Ini membantu menghindari kesalahan dan memastikan bahwa metode yang ditulis benar-benar menggantikan metode di kelas dasar.

\textbf{Contoh Override:}

\begin{lstlisting}[style=JavaStyle]
	@Override
	public double getArea() {
		return width * height;
	}
\end{lstlisting}

Pada contoh di atas, metode \texttt{getArea()} yang dideklarasikan dalam kelas \texttt{Rectangle} menggantikan metode \texttt{getArea()} yang dideklarasikan dalam kelas \texttt{Shape}. Dengan menggunakan \texttt{@Override}, kita menandakan bahwa metode ini menggantikan implementasi metode yang sama dari kelas dasar, sehingga meningkatkan kejelasan dan mencegah potensi kesalahan.

\subsection{Interface}

\textbf{Interface} adalah sebuah kontrak dalam pemrograman berorientasi objek yang mendefinisikan metode tanpa memberikan implementasinya. Interface berguna untuk mendefinisikan perilaku umum yang harus dimiliki oleh berbagai kelas tanpa memerlukan pewarisan langsung. Setiap kelas yang mengimplementasikan sebuah interface harus menyediakan implementasi untuk semua metode yang didefinisikan dalam interface tersebut.

\textbf{Contoh Interface:}

\begin{lstlisting}[style=JavaStyle, caption={Drawable.java}]
	package edu.example;
	
	public interface Drawable {
		void draw();
	}
\end{lstlisting}

Pada contoh di atas, \texttt{Drawable} adalah interface dengan metode \texttt{draw()} yang harus diimplementasikan oleh kelas apa pun yang mengimplementasikan interface ini. Interface ini memungkinkan berbagai kelas untuk memiliki metode \texttt{draw()} dengan cara yang sesuai dengan kelas masing-masing.

\subsection{Implementasi Kelas \texttt{Rectangle}}

Kelas \texttt{Rectangle} adalah salah satu kelas turunan dari \texttt{Shape} yang juga mengimplementasikan interface \texttt{Drawable}. Kelas ini memiliki atribut untuk lebar (\texttt{width}) dan tinggi (\texttt{height}) serta metode untuk menghitung luas dan menggambar persegi panjang.

\begin{lstlisting}[style=JavaStyle]
	package edu.example;
	
	public class Rectangle extends Shape implements Drawable {
		private double width;
		private double height;
		
		public Rectangle(String color, double width, double height) {
			super(color);
			this.width = width;
			this.height = height;
		}
		
		@Override
		public double getArea() {
			return width * height;
		}
		
		@Override
		public void draw() {
			System.out.println("Drawing a rectangle with width " + width + " and height " + height);
		}
	}
\end{lstlisting}

Pada contoh di atas, kelas \texttt{Rectangle} mengimplementasikan interface \texttt{Drawable} dan menyediakan implementasi untuk metode \texttt{draw()}. Selain itu, kelas ini mengoverride metode abstrak \texttt{getArea()} dari \texttt{Shape} untuk menghitung luas persegi panjang.

\subsection{Implementasi Kelas \texttt{Triangle}}

Sebagai tambahan dari \texttt{Rectangle}, kita juga dapat membuat kelas \texttt{Triangle} yang mengimplementasikan interface \texttt{Drawable} dan mewarisi kelas abstrak \texttt{Shape}. Kelas \texttt{Triangle} ini akan memiliki metode untuk menghitung luas dan menggambar segitiga.

\begin{lstlisting}[style=JavaStyle]
	package edu.example;
	
	public class Triangle extends Shape implements Drawable {
		private double base;
		private double height;
		
		public Triangle(String color, double base, double height) {
			super(color);
			this.base = base;
			this.height = height;
		}
		
		@Override
		public double getArea() {
			return 0.5 * base * height;
		}
		
		@Override
		public void draw() {
			System.out.println("Drawing a triangle with base " + base + " and height " + height);
		}
	}
\end{lstlisting}

Pada contoh di atas, kelas \texttt{Triangle} mengimplementasikan interface \texttt{Drawable} dan menyediakan implementasi untuk metode \texttt{draw()}. Selain itu, kelas ini mengoverride metode abstrak \texttt{getArea()} untuk menghitung luas segitiga.

\subsection{Demonstrasi Polimorfisme dengan \texttt{Rectangle} dan \texttt{Triangle}}

Polimorfisme memungkinkan kita untuk menggunakan objek \texttt{Rectangle} dan \texttt{Triangle} secara seragam melalui referensi kelas abstrak \texttt{Shape} atau interface \texttt{Drawable}. Berikut adalah contoh kode yang menunjukkan cara menggunakan polimorfisme dalam program untuk menggambar dan menghitung luas bentuk-bentuk ini.

\begin{lstlisting}[style=JavaStyle]
	package edu.example;
	
	public class ShapeDemo {
		public static void main(String[] args) {
			Shape[] shapes = {
				new Rectangle("blue", 4, 5),
				new Triangle("green", 3, 6)
			};
			
			for (Shape shape : shapes) {
				System.out.println("Color: " + shape.getColor());
				System.out.println("Area: " + shape.getArea());
				
				if (shape instanceof Drawable) {
					((Drawable) shape).draw();
				}
				
				System.out.println();
			}
		}
	}
\end{lstlisting}

\textbf{Penjelasan Kode:}
Pada contoh di atas, array \texttt{shapes} berisi objek \texttt{Rectangle} dan \texttt{Triangle}. Dengan menggunakan referensi \texttt{Shape}, kita dapat memanggil metode \texttt{getColor()} dan \texttt{getArea()} untuk masing-masing objek tanpa harus mengetahui tipe spesifiknya. Selain itu, kita menggunakan polimorfisme dengan interface \texttt{Drawable} untuk memanggil metode \texttt{draw()} pada objek yang mengimplementasikan interface ini.


\subsection{Implementasi dan Penggunaan}

Kelas abstrak, pewarisan, dan \texttt{override} digunakan untuk membangun struktur hierarki yang lebih kompleks dengan kode yang dapat digunakan kembali. Kelas turunan dapat memperluas fungsionalitas kelas dasar, menyesuaikan perilaku metode, dan memperbaiki implementasi metode abstrak sesuai dengan kebutuhan aplikasi.

\section{Contoh Implementasi Kode Abstrak dan Pewarisan di Java}

\subsection{Kelas \texttt{Question}}

\begin{lstlisting}[style=JavaStyle]
	package edu.pradita;
	
	public abstract class Question {
		
		private String text;
		private Object correctAnswer;
		
		public Question(String text, Object correctAnswer) {
			this.text = text;
			this.correctAnswer = correctAnswer;
		}
		
		public void display() {
			System.out.println(this.getText());
		}
		
		public boolean checkSubmittedAnswer(Object submittedAnswer) {
			if (correctAnswer.equals(submittedAnswer)) {
				return true;
			} else {
				return false;
			}
		}
		
		public String getText() {
			return text;
		}
		
		public Object getCorrectAnswer() {
			return correctAnswer;
		}
	}
\end{lstlisting}

\textbf{Penjelasan:} Kelas \texttt{Question} adalah kelas abstrak yang menyimpan informasi umum tentang pertanyaan, termasuk teks pertanyaan dan jawaban yang benar. Metode \texttt{display()} menampilkan teks pertanyaan, dan metode \texttt{checkSubmittedAnswer(Object submittedAnswer)} memeriksa apakah jawaban yang diberikan sesuai dengan jawaban yang benar.

\subsection{Kelas \texttt{OpenQuestion}}

\begin{lstlisting}[style=JavaStyle]
	package edu.pradita;
	
	public class OpenQuestion extends Question {
		
		public OpenQuestion(String text) {
			super(text, null);
		}
		
		@Override
		public boolean checkSubmittedAnswer(Object submittedAnswer) {
			return true;
		}
	}
\end{lstlisting}

\textbf{Penjelasan:} Kelas \texttt{OpenQuestion} adalah turunan dari kelas \texttt{Question} untuk pertanyaan terbuka. Konstruktor hanya memerlukan teks pertanyaan dan tidak memerlukan jawaban yang benar. Metode \texttt{checkSubmittedAnswer(Object submittedAnswer)} selalu mengembalikan \texttt{true} karena semua jawaban dianggap benar.

\subsection{Kelas \texttt{NumericQuestion}}

\begin{lstlisting}[style=JavaStyle]
	package edu.pradita;
	
	public class NumericQuestion extends Question {
		
		public NumericQuestion(String text, int correctAnswer) {
			super(text, correctAnswer);
		}
		
		public NumericQuestion(String text, double correctAnswer) {
			super(text, correctAnswer);
		}
		
		@Override
		public boolean checkSubmittedAnswer(Object submittedAnswer) {
			return (double) this.getCorrectAnswer() == Double.valueOf(submittedAnswer.toString());
		}
	}
\end{lstlisting}

\textbf{Penjelasan:} Kelas \texttt{NumericQuestion} menangani pertanyaan dengan jawaban numerik. Terdapat dua konstruktor untuk menerima jawaban yang benar bertipe \texttt{int} atau \texttt{double}. Metode \texttt{checkSubmittedAnswer(Object submittedAnswer)} memeriksa apakah jawaban yang diberikan sesuai dengan jawaban numerik yang benar.

\subsection{Kelas \texttt{FilledInQuestion}}

\begin{lstlisting}[style=JavaStyle]
	package edu.pradita;
	
	public class FilledInQuestion extends Question {
		
		public FilledInQuestion(String text, String correctAnswer) {
			super(text, correctAnswer);
		}
	}
\end{lstlisting}

\textbf{Penjelasan:} Kelas \texttt{FilledInQuestion} menangani pertanyaan dengan jawaban yang harus diisi. Konstruktor menerima teks pertanyaan dan jawaban yang benar bertipe \texttt{String}.

\subsection{Kelas \texttt{MultipleChoiceQuestion}}

\begin{lstlisting}[style=JavaStyle]
	package edu.pradita;
	
	import java.util.ArrayList;
	import java.util.List;
	
	public class MultipleChoiceQuestion extends Question {
		
		private List<Object> choices = new ArrayList<>();
		
		public MultipleChoiceQuestion(String text, Object correctAnswer, List<Object> choices) {
			super(text, correctAnswer);
			this.choices.addAll(choices);
		}
		
		@Override
		public void display() {
			display(true);
		}
		
		public void display(boolean withAbc) {
			super.display();
			int code = 97;
			for (Object choice : choices) {
				char head = '-';
				if (withAbc) {
					head = (char) code;
				}
				System.out.println(head + " " + choice);
				code++;
			}
		}
		
		public List<Object> getChoices() {
			return choices;
		}
	}
\end{lstlisting}

\textbf{Penjelasan:} Kelas \texttt{MultipleChoiceQuestion} menangani pertanyaan pilihan ganda. Selain teks pertanyaan dan jawaban yang benar, kelas ini juga menyimpan daftar pilihan jawaban. Metode \texttt{display()} menampilkan pertanyaan dan opsi jawaban dengan atau tanpa huruf ABC.

\subsection{Kelas \texttt{TrueFalseQuestion}}

\begin{lstlisting}[style=JavaStyle]
	package edu.pradita;
	
	import java.util.ArrayList;
	import java.util.Arrays;
	
	public class TrueFalseQuestion extends MultipleChoiceQuestion {
		
		public TrueFalseQuestion(String text, Object correctAnswer) {
			super(text, true, new ArrayList<>(Arrays.asList(true, false)));
		}
		
		@Override
		public boolean checkSubmittedAnswer(Object submittedAnswer) {
			return (boolean) this.getCorrectAnswer() == Boolean.valueOf(submittedAnswer.toString());
		}
	}
\end{lstlisting}

\textbf{Penjelasan:} Kelas \texttt{TrueFalseQuestion} adalah turunan dari \texttt{MultipleChoiceQuestion} khusus untuk pertanyaan benar/salah. Konstruktornya menerima teks pertanyaan dan jawaban yang benar, dan selalu menyertakan pilihan \texttt{true} dan \texttt{false}.

\subsection{Kelas \texttt{Main}}

\begin{lstlisting}[style=JavaStyle]
	package edu.pradita;
	
	import java.util.ArrayList;
	import java.util.Arrays;
	import java.util.List;
	import java.util.Scanner;
	
	public class Main {
		
		public static void main(String[] args) {
			
			List<Question> questions = new ArrayList<>();
			
			questions.add(new FilledInQuestion("He is __ farmer.", "a"));
			questions.add(new NumericQuestion("1 + 1 = ?", 2));
			questions.add(new OpenQuestion("What is your plan for the next semester?"));
			questions.add(new MultipleChoiceQuestion("_____ in the Wonderland.",
			"Alice",
			new ArrayList<>(Arrays.asList("Alice", "Bob", "Charlie"))
			));
			questions.add(new TrueFalseQuestion("Is today Tuesday?", true));
			
			Scanner scanner = new Scanner(System.in);
			
			for (int lineNum = 1; lineNum <= questions.size(); lineNum++) {
				Question question = questions.get(lineNum - 1);
				System.out.println("Question " + lineNum + ":");
				question.display();
				System.out.print("Your answer: ");
				String answer = scanner.nextLine().trim();
				boolean result = question.checkSubmittedAnswer(answer);
				System.out.println("Result: " + result);
				System.out.println();
			}
			
			scanner.close();
		}
	}
\end{lstlisting}

\textbf{Penjelasan:} Kelas \texttt{Main} adalah kelas utama yang menampilkan dan memproses daftar pertanyaan. Program membuat beberapa pertanyaan dari berbagai jenis, menambahkannya ke dalam daftar, dan kemudian menampilkan setiap pertanyaan satu per satu. Pengguna diminta untuk memberikan jawaban, yang kemudian diperiksa dengan metode \texttt{checkSubmittedAnswer()}. Hasil dari setiap jawaban ditampilkan setelah input.

\section{Latihan}


Kerjakanlah contoh-contoh kasus berikut! Ketik kode program dan jalankan! Amati struktur, output, dan perilaku program. Cobalah menjawab pertanyaan-pertanyaan yang diberikan untuk meningkatkan pemahaman mengenai konsep yang diajarkan dengan mengisi persegi kosong yang sediakan.

\subsection{Contoh Sistem Pembayaran Online}

\subsubsection{Langkah 1: Kelas Abstrak dan Pewarisan}

\textbf{Kode:} Mendefinisikan kelas abstrak \texttt{Payment} dan kelas konkret \texttt{CreditCardPayment} dan \texttt{PayPalPayment}.

\begin{lstlisting}[style=JavaStyle, caption={Payment.java}]
	package com.example.payment;
	
	public abstract class Payment {
		double amount;
		
		public Payment(double amount) {
			this.amount = amount;
		}
		
		public abstract void processPayment();
		
		public void displayAmount() {
			System.out.println("Payment Amount: $" + amount);
		}
	}
\end{lstlisting}

\begin{lstlisting}[style=JavaStyle, caption={CreditCardPayment.java}]
	package com.example.payment.types;
	
	import com.example.payment.Payment;
	
	public class CreditCardPayment extends Payment {
		public CreditCardPayment(double amount) {
			super(amount);
		}
		
		@Override
		public void processPayment() {
			System.out.println("Processing credit card payment of $" + amount);
		}
	}
\end{lstlisting}

\begin{lstlisting}[style=JavaStyle, caption={PayPalPayment.java}]
	package com.example.payment.types;
	
	import com.example.payment.Payment;
	
	public class PayPalPayment extends Payment {
		public PayPalPayment(double amount) {
			super(amount);
		}
		
		@Override
		public void displayAmount() {
			System.out.println("Payment Amount: USD" + amount);
		}
		
		@Override
		public void processPayment() {
			System.out.println("Processing PayPal payment of USD" + amount);
		}
	}
\end{lstlisting}

\textbf{Pertanyaan Refleksi:}
\begin{enumerate}
	\item Cobalah buat object dari kelas abstrak dalam metode \texttt{main}! Coba jalankan program. Bagaimankah hasilnya? Mengapa demikian?
	\begin{lstlisting}[style=JavaStyle, caption={TestAbstractClassInstantiation.java}]
		package com.example.payment.system;
		
		import com.example.payment.Payment;
		
		public class TestAbstractClassInstantiation {
			public static void main(String[] args) {
				Payment payment = new Payment(100.0);
				System.out.println("End");
			}
		}
	\end{lstlisting}
	\begin{tcolorbox}[colback=white, colframe=black,  width=\linewidth, height=3cm,  boxrule=1pt, sharp corners]
	\end{tcolorbox}
	\item Coba buat objek dari kelas-kelas yang di-\texttt{extends} dari kelas abstrak? Coba jalankan program. Bagaimankah hasilnya? Mengapa demikian?
	\begin{lstlisting}[style=JavaStyle, caption={TestSubclassInstantiation.java}]
		package com.example.payment.system;
		
		import com.example.payment.Payment;
		import com.example.payment.types.CreditCardPayment;
		import com.example.payment.types.PayPalPayment;
		
		public class TestSubclassInstantiation {
			public static void main(String[] args) {
				// Membuat objek dari kelas-kelas turunan
				Payment creditCardPayment = new CreditCardPayment(200.0);
				Payment payPalPayment = new PayPalPayment(150.0);
				System.out.println("End");
			}
		}
	\end{lstlisting}
	\begin{tcolorbox}[colback=white, colframe=black,  width=\linewidth, height=3cm,  boxrule=1pt, sharp corners]
	\end{tcolorbox}
	\item Coba hapus metode \texttt{processPayment()} dari kelas-kelas yang di-extends dari kelas abstrak. Coba jalankan program. Bagaimankah hasilnya? Mengapa demikian?
	\begin{lstlisting}[style=JavaStyle, caption={TestSubclassInstantiation.java}]
		package com.example.payment.system;
		
		import com.example.payment.Payment;
		import com.example.payment.types.CreditCardPayment;
		import com.example.payment.types.PayPalPayment;
		
		public class TestSubclassInstantiation {
			public static void main(String[] args) {
				// Membuat objek dari kelas-kelas turunan
				Payment creditCardPayment = new CreditCardPayment(200.0);
				Payment payPalPayment = new PayPalPayment(150.0);
				
				// Memproses setiap pembayaran
				creditCardPayment.processPayment();
				creditCardPayment.displayAmount();
				
				System.out.println();
				
				payPalPayment.processPayment();
				payPalPayment.displayAmount();
				
				System.out.println("End");
			}
		}
	\end{lstlisting}
	\begin{tcolorbox}[colback=white, colframe=black,  width=\linewidth, height=3cm,  boxrule=1pt, sharp corners]
	\end{tcolorbox}
	\item Mengapa \texttt{creditCardPayment} dapat menampilkan \texttt{Payment Amount: \$} walaupun metode untuk menampilkan teks tersebut TIDAK terdapat dalam kelas \texttt{creditCardPayment}?
	\begin{tcolorbox}[colback=white, colframe=black,  width=\linewidth, height=3cm,  boxrule=1pt, sharp corners]
	\end{tcolorbox}
	\item Mengapa objek \texttt{paypalPayment} menampilkan \texttt{Payment Amount: \textbf{USD}} sedangkan \texttt{creditCardPayment} menampikan \texttt{Payment Amount: \textbf{\$}}?
	\begin{tcolorbox}[colback=white, colframe=black,  width=\linewidth, height=3cm,  boxrule=1pt, sharp corners]
	\end{tcolorbox}
	\item Mengapa \texttt{Payment} didefinisikan sebagai kelas abstrak daripada kelas biasa?
	\begin{tcolorbox}[colback=white, colframe=black,  width=\linewidth, height=3cm,  boxrule=1pt, sharp corners]
	\end{tcolorbox}
	\item Bagaimana pewarisan dari \texttt{Payment} ke \texttt{CreditCardPayment} dan \texttt{PayPalPayment} mengurangi redundansi?
	\begin{tcolorbox}[colback=white, colframe=black,  width=\linewidth, height=3cm,  boxrule=1pt, sharp corners]
	\end{tcolorbox}
	\item Apa manfaat dari struktur ini jika kita ingin menambahkan metode pembayaran baru di masa depan?
	\begin{tcolorbox}[colback=white, colframe=black,  width=\linewidth, height=3cm,  boxrule=1pt, sharp corners]
	\end{tcolorbox}
\end{enumerate}

\subsubsection{Langkah 2: Implementasi Interface}

\textbf{Kode:} Mengimplementasikan antarmuka \texttt{Refundable} untuk \texttt{DebitCardPayment}.

\begin{lstlisting}[style=JavaStyle, caption={Refundable.java}]
	package com.example.payment.features;
	
	public interface Refundable {
		void refund();
	}
\end{lstlisting}

\begin{lstlisting}[style=JavaStyle, caption={DebitCardPayment.java}]
	package com.example.payment.types;
	
	import com.example.payment.Payment;
	import com.example.payment.features.Refundable;
	
	public class DebitCardPayment extends Payment implements Refundable {
		public DebitCardPayment(double amount) {
			super(amount);
		}
		
		@Override
		public void processPayment() {
			System.out.println("Processing debit card payment of $" + amount);
		}
		
		@Override
		public void refund() {
			System.out.println("Refunding debit card payment of $" + amount);
		}
	}
\end{lstlisting}

\begin{lstlisting}[style=JavaStyle, caption={PaymentApp.java}]
	package com.example.payment;
	
	import com.example.payment.types.DebitCardPayment;
	
	public class PaymentApp {
		public static void main(String[] args) {
			// Membuat objek dari kelas DebitCardPayment
			DebitCardPayment debitCardPayment = new DebitCardPayment(100.0);
			
			// Memproses pembayaran
			debitCardPayment.processPayment();
			
			// Melakukan pengembalian dana (refund)
			debitCardPayment.refund();
		}
	}
\end{lstlisting}

\textbf{Pertanyaan Refleksi:}
\begin{enumerate}
	\item Coba hapus metode \texttt{refund()} yang ada di dalam kelas yang mengimplementasikan (\texttt{implements}) \texttt{Refundable}, bukan yang ada di dalam \texttt{interface}? Coba jalankan program. Apa yang terjadi? Mengapa demikian?
	\begin{tcolorbox}[colback=white, colframe=black,  width=\linewidth, height=3cm,  boxrule=1pt, sharp corners]
	\end{tcolorbox}
	\item Bagaimana penerapan antarmuka \texttt{Refundable} memengaruhi kelas \texttt{DebitCardPayment}?
	\begin{tcolorbox}[colback=white, colframe=black,  width=\linewidth, height=3cm,  boxrule=1pt, sharp corners]
	\end{tcolorbox}
	\item Apa keuntungan dari menggunakan antarmuka seperti \texttt{Refundable} dalam sistem pembayaran?
	\begin{tcolorbox}[colback=white, colframe=black,  width=\linewidth, height=3cm,  boxrule=1pt, sharp corners]
	\end{tcolorbox}
	\item Bagaimana desain ini memudahkan penambahan tipe pembayaran baru yang mendukung pengembalian dana di masa depan?
	\begin{tcolorbox}[colback=white, colframe=black,  width=\linewidth, height=3cm,  boxrule=1pt, sharp corners]
	\end{tcolorbox}
\end{enumerate}

\subsubsection{Langkah 3: Demonstrasi Polimorfisme}

Ubah 2 kelas \texttt{*Payment} sebelumnya dengan mengimplementasikan interface \texttt{Refundable} pada kedua kelas tersebut! 

\begin{lstlisting}[style=JavaStyle, caption={CreditCardPayment.java}]
	package com.example.payment.types;
	
	import com.example.payment.Payment;
	import com.example.payment.features.Refundable;
	
	public class CreditCardPayment extends Payment implements Refundable {
		public CreditCardPayment(double amount) {
			super(amount);
		}
		
		@Override
		public void processPayment() {
			System.out.println("Processing credit card payment of $" + amount);
		}
		
		@Override
		public void refund() {
			System.out.println("Refunding credit card payment of $" + amount);
		}
	}
\end{lstlisting}

\begin{lstlisting}[style=JavaStyle, caption={PayPalPayment.java}]
	package com.example.payment.types;
	
	import com.example.payment.Payment;
	import com.example.payment.features.Refundable;
	
	public class PayPalPayment extends Payment implements Refundable {
		public PayPalPayment(double amount) {
			super(amount);
		}
		
		@Override
		public void displayAmount() {
			System.out.println("Payment Amount: USD" + amount);
		}
		
		@Override
		public void processPayment() {
			System.out.println("Processing PayPal payment of USD" + amount);
		}
		
		@Override
		public void refund() {
			System.out.println("Refunding PayPal payment of USD" + amount);
		}
	}
\end{lstlisting}

Buat dan jalankan kode berikut kelas \texttt{PaymentSystem}. 

\begin{lstlisting}[style=JavaStyle, caption={PaymentSystem.java}]
	package com.example.payment.system;
	
	import com.example.payment.Payment;
	import com.example.payment.features.Refundable;
	import com.example.payment.types.CreditCardPayment;
	import com.example.payment.types.PayPalPayment;
	import com.example.payment.types.DebitCardPayment;
	
	public class PaymentSystem {
		public static void main(String[] args) {
			Payment[] payments = {
				new CreditCardPayment(100),
				new PayPalPayment(150),
				new DebitCardPayment(200)
			};
			
			for (Payment payment : payments) {
				payment.processPayment();
				if (payment instanceof Refundable) {
					((Refundable) payment).refund();
				}
			}
		}
	}
\end{lstlisting}

\textbf{Pertanyaan Refleksi:}
\begin{enumerate}
	\item Mengapa masing-masing objek menampilkan output yang berbeda walaupun mereka sama-sama mengeksekusi metode \texttt{refund}?
	\begin{tcolorbox}[colback=white, colframe=black,  width=\linewidth, height=3cm,  boxrule=1pt, sharp corners]
	\end{tcolorbox}
	\item Bagaimana polimorfisme memungkinkan kita untuk menangani berbagai jenis pembayaran menggunakan referensi \texttt{Payment} tunggal?
	\begin{tcolorbox}[colback=white, colframe=black,  width=\linewidth, height=3cm,  boxrule=1pt, sharp corners]
	\end{tcolorbox}
	\item Mengapa polimorfisme bermanfaat untuk menangani berbagai metode pembayaran dengan cara yang lebih fleksibel?
	\begin{tcolorbox}[colback=white, colframe=black,  width=\linewidth, height=3cm,  boxrule=1pt, sharp corners]
	\end{tcolorbox}
	\item Jelaskan bagaimana desain ini akan mendukung penambahan tipe pembayaran tambahan dengan perubahan kode minimal.
	\begin{tcolorbox}[colback=white, colframe=black,  width=\linewidth, height=3cm,  boxrule=1pt, sharp corners]
	\end{tcolorbox}
\end{enumerate}



\subsubsection{Langkah 4: Menyimpulkan}
Dari eksperimen di atas, buatlah pengertian dan manfaat dari konsep-konsep berikut dalam konteks pemrograman berorientasi objek dengan menggunakan kata-kata Anda sendiri.
\begin{enumerate}
	\item Kelas Abstrak
	\begin{tcolorbox}[colback=white, colframe=black,  width=\linewidth, height=3cm,  boxrule=1pt, sharp corners]
	\end{tcolorbox}
	\item Pewarisan/\textit{Inheritance} (\texttt{extends})
	\begin{tcolorbox}[colback=white, colframe=black,  width=\linewidth, height=3cm,  boxrule=1pt, sharp corners]
	\end{tcolorbox}
	\item Overriding
	\begin{tcolorbox}[colback=white, colframe=black,  width=\linewidth, height=3cm,  boxrule=1pt, sharp corners]
	\end{tcolorbox}
	\item Interface
	\begin{tcolorbox}[colback=white, colframe=black,  width=\linewidth, height=3cm,  boxrule=1pt, sharp corners]
	\end{tcolorbox}
	\item Polimorfisme
	\begin{tcolorbox}[colback=white, colframe=black,  width=\linewidth, height=3cm,  boxrule=1pt, sharp corners]
	\end{tcolorbox}
\end{enumerate}


\section{Soal}

Kerjakanlah contoh-contoh kasus berikut! Ketik kode program dan jalankan! Amati struktur, output, dan perilaku program. Cobalah menjawab pertanyaan-pertanyaan yang diberikan untuk meningkatkan pemahaman mengenai konsep yang diajarkan dengan mengisi persegi kosong yang sediakan.

\subsection{Contoh Sistem Pemesanan Transportasi}

\subsubsection{Langkah 1: Kelas Abstrak dan Pewarisan}

\textbf{Kode:} Mendefinisikan kelas abstrak \texttt{Vehicle} dan kelas konkret \texttt{Car} serta \texttt{Bus}.

\begin{lstlisting}[style=JavaStyle, caption={Vehicle.java}]
	package com.example.transport;
	
	public abstract class Vehicle {
		String id;
		
		public Vehicle(String id) {
			this.id = id;
		}
		
		public abstract double calculateFare(double distance);
		
		public void displayType() {
			System.out.println("Vehicle ID: " + id);
		}
	}
\end{lstlisting}

\begin{lstlisting}[style=JavaStyle, caption={Car.java}]
	package com.example.transport.types;
	
	import com.example.transport.Vehicle;
	
	public class Car extends Vehicle {
		public Car(String id) {
			super(id);
		}
		
		@Override
		public double calculateFare(double distance) {
			return distance * 0.5;
		}
	}
\end{lstlisting}

\begin{lstlisting}[style=JavaStyle, caption={Bus.java}]
	package com.example.transport.types;
	
	import com.example.transport.Vehicle;
	
	public class Bus extends Vehicle {
		public Bus(String id) {
			super(id);
		}
		
		@Override
		public double calculateFare(double distance) {
			return distance * 0.2;
		}
	}
\end{lstlisting}

\textbf{Pertanyaan Refleksi:}
\begin{enumerate}
	\item Cobalah buat objek dari kelas abstrak dalam metode \texttt{main}! Apa hasilnya? Mengapa demikian?
	\begin{tcolorbox}[colback=white, colframe=black,  width=\linewidth, height=3cm, boxrule=1pt, sharp corners]
	\end{tcolorbox}
	\item Coba buat objek dari kelas-kelas turunan dari kelas abstrak \texttt{Vehicle}. Apa hasilnya? Mengapa demikian?
	\begin{tcolorbox}[colback=white, colframe=black,  width=\linewidth, height=3cm, boxrule=1pt, sharp corners]
	\end{tcolorbox}
	\item Bagaimana mendefinisikan \texttt{Vehicle} sebagai kelas abstrak membantu mengelompokkan atribut dan metode umum untuk semua tipe kendaraan?
	\begin{tcolorbox}[colback=white, colframe=black, width=\linewidth, height=3cm, boxrule=1pt, sharp corners]
	\end{tcolorbox}
	\item Apa manfaat pewarisan saat mengimplementasikan tipe kendaraan yang berbeda seperti \texttt{Car} dan \texttt{Bus}?
	\begin{tcolorbox}[colback=white, colframe=black, width=\linewidth, height=3cm, boxrule=1pt, sharp corners]
	\end{tcolorbox}
	\item Dalam skenario apa Anda mempertimbangkan untuk menggunakan pola ini dalam aplikasi lain?
	\begin{tcolorbox}[colback=white, colframe=black, width=\linewidth, height=3cm, boxrule=1pt, sharp corners]
	\end{tcolorbox}
\end{enumerate}

\subsubsection{Langkah 2: Implementasi Interface}

\textbf{Kode:} Mengimplementasikan antarmuka \texttt{Trackable} untuk \texttt{Train}.

\begin{lstlisting}[style=JavaStyle, caption={Trackable.java}]
	package com.example.transport.features;
	
	public interface Trackable {
		void getLocation();
	}
\end{lstlisting}

\begin{lstlisting}[style=JavaStyle, caption={Train.java}]
	package com.example.transport.types;
	
	import com.example.transport.Vehicle;
	import com.example.transport.features.Trackable;
	
	public class Train extends Vehicle implements Trackable {
		public Train(String id) {
			super(id);
		}
		
		@Override
		public double calculateFare(double distance) {
			return distance * 0.3;
		}
		
		@Override
		public void getLocation() {
			System.out.println("Tracking train location for ID: " + id);
		}
	}
\end{lstlisting}

\textbf{Pertanyaan Refleksi:}
\begin{enumerate}
	\item Bagaimana antarmuka \texttt{Trackable} memungkinkan \texttt{Train} memiliki perilaku unik tanpa mengubah kelas \texttt{Vehicle}?
	\begin{tcolorbox}[colback=white, colframe=black,  width=\linewidth, height=3cm, boxrule=1pt, sharp corners]
	\end{tcolorbox}
	\item Mengapa memisahkan perilaku pelacakan lokasi ke dalam antarmuka merupakan hal yang bermanfaat?
	\begin{tcolorbox}[colback=white, colframe=black,  width=\linewidth, height=3cm, boxrule=1pt, sharp corners]
	\end{tcolorbox}
	\item Bagaimana Anda akan menggunakan antarmuka \texttt{Trackable} untuk tipe kendaraan lain yang memerlukan pelacakan?
	\begin{tcolorbox}[colback=white, colframe=black,  width=\linewidth, height=3cm, boxrule=1pt, sharp corners]
	\end{tcolorbox}
\end{enumerate}

\subsubsection{Langkah 3: Demonstrasi Polimorfisme}

\textbf{Kode:} Menunjukkan perilaku polimorfisme dengan kelas \texttt{BookingSystem}.

\begin{lstlisting}[style=JavaStyle, caption={BookingSystem.java}]
	package com.example.transport.system;
	
	import com.example.transport.Vehicle;
	import com.example.transport.features.Trackable;
	import com.example.transport.types.Car;
	import com.example.transport.types.Bus;
	import com.example.transport.types.Train;
	
	public class BookingSystem {
		public static void main(String[] args) {
			Vehicle[] vehicles = {
				new Car("C123"),
				new Bus("B456"),
				new Train("T789")
			};
			
			for (Vehicle vehicle : vehicles) {
				vehicle.displayType();
				System.out.println("Fare for 100 miles: $" + vehicle.calculateFare(100));
				if (vehicle instanceof Trackable) {
					((Trackable) vehicle).getLocation();
				}
			}
		}
	}
\end{lstlisting}

\textbf{Pertanyaan Refleksi:}
\begin{enumerate}
	\item Bagaimana polimorfisme memungkinkan kita menggunakan referensi \texttt{Vehicle} untuk mengelola berbagai jenis kendaraan?
	\begin{tcolorbox}[colback=white, colframe=black,  width=\linewidth, height=3cm, boxrule=1pt, sharp corners]
	\end{tcolorbox}
	\item Fleksibilitas apa yang ditambahkan oleh desain ini jika kita perlu menambahkan tipe kendaraan baru, seperti \texttt{Taxi}?
	\begin{tcolorbox}[colback=white, colframe=black,  width=\linewidth, height=3cm, boxrule=1pt, sharp corners]
	\end{tcolorbox}
	\item Mengapa polimorfisme berguna dalam sistem seperti sistem pemesanan transportasi?
	\begin{tcolorbox}[colback=white, colframe=black,  width=\linewidth, height=3cm, boxrule=1pt, sharp corners]
	\end{tcolorbox}
\end{enumerate}

%
%\subsection{Contoh Sistem Manajemen Karyawan}
%
%\subsubsection{Langkah 1: Kelas Abstrak dan Pewarisan}
%
%\textbf{Kode:} Mendefinisikan kelas abstrak \texttt{Employee} dan membuat subclass untuk \texttt{FullTimeEmployee} dan \texttt{PartTimeEmployee}.
%
%\begin{lstlisting}[style=JavaStyle, caption={Employee.java}]
%	package com.example.employee;
%	
%	public abstract class Employee {
	%		String name;
	%		
	%		public Employee(String name) {
		%			this.name = name;
		%		}
	%		
	%		public abstract double calculateCompensation();
	%		
	%		public void displayEmployee() {
		%			System.out.println("Employee Name: " + name);
		%		}
	%	}
%\end{lstlisting}
%
%\begin{lstlisting}[style=JavaStyle, caption={FullTimeEmployee.java dan PartTimeEmployee.java}]
%	package com.example.employee.types;
%	
%	import com.example.employee.Employee;
%	
%	public class FullTimeEmployee extends Employee {
	%		public FullTimeEmployee(String name) {
		%			super(name);
		%		}
	%		
	%		@Override
	%		public double calculateCompensation() {
		%			return 5000; // Gaji tetap
		%		}
	%	}
%	
%	public class PartTimeEmployee extends Employee {
	%		public PartTimeEmployee(String name) {
		%			super(name);
		%		}
	%		
	%		@Override
	%		public double calculateCompensation() {
		%			return 3000; // Gaji tetap
		%		}
	%	}
%\end{lstlisting}
%
%\textbf{Pertanyaan Refleksi:}
%\begin{itemize}
%	\item Cobalah buat object dari kelas abstrak dalam metode \texttt{main}! Apakah bisa atau tidak? Jika tidak bisa, mengapa?
%	\item Cobat buat objek dari kelas-kelas yang diturun dari kelas abstrak? Apakah bisa atau tidak? Jika bisa, mengapa?
%	\item Bagaimana menggunakan kelas abstrak \texttt{Employee} membantu standarisasi perilaku karyawan?
%	\item Apa manfaat pewarisan saat mendefinisikan tipe karyawan berbeda dengan metode kompensasi unik?
%	\item Dalam konteks apa lagi desain ini bisa diterapkan?
%\end{itemize}
%
%\subsubsection{Langkah 2: Implementasi Interface}
%
%\textbf{Kode:} Mengimplementasikan antarmuka \texttt{Promotable} untuk kelas \texttt{Contractor}.
%
%\begin{lstlisting}[style=JavaStyle, caption={Promotable.java}]
%	package com.example.employee.features;
%	
%	public interface Promotable {
	%		void promote();
	%	}
%\end{lstlisting}
%
%\begin{lstlisting}[style=JavaStyle, caption={Contractor.java}]
%	package com.example.employee.types;
%	
%	import com.example.employee.Employee;
%	import com.example.employee.features.Promotable;
%	
%	public class Contractor extends Employee implements Promotable {
	%		public Contractor(String name) {
		%			super(name);
		%		}
	%		
	%		@Override
	%		public double calculateCompensation() {
		%			return 4000; // Gaji tetap
		%		}
	%		
	%		@Override
	%		public void promote() {
		%			System.out.println("Promoting contractor: " + name);
		%		}
	%	}
%\end{lstlisting}
%
%\textbf{Pertanyaan Refleksi:}
%\begin{itemize}
%	\item Apa keuntungan dari antarmuka \texttt{Promotable} bagi kelas \texttt{Contractor}?
%	\item Mengapa kita memilih menggunakan antarmuka daripada menambahkan metode \texttt{promote} langsung ke \texttt{Employee}?
%	\item Bagaimana pendekatan ini memudahkan penerapan promosi untuk tipe karyawan tertentu?
%\end{itemize}
%
%\subsubsection{Langkah 3: Demonstrasi Polimorfisme}
%
%\textbf{Kode:} Menunjukkan perilaku polimorfisme dengan kelas \texttt{EmployeeManagementSystem}.
%
%\begin{lstlisting}[style=JavaStyle, caption={EmployeeManagementSystem.java}]
%	package com.example.employee.system;
%	
%	import com.example.employee.Employee;
%	import com.example.employee.features.Promotable;
%	import com.example.employee.types.FullTimeEmployee;
%	import com.example.employee.types.PartTimeEmployee;
%	import com.example.employee.types.Contractor;
%	
%	public class EmployeeManagementSystem {
	%		public static void main(String[] args) {
		%			Employee[] employees = {
			%				new FullTimeEmployee("Alice"),
			%				new PartTimeEmployee("Bob"),
			%				new Contractor("Charlie")
			%			};
		%			
		%			for (Employee employee : employees) {
			%				employee.displayEmployee();
			%				System.out.println("Compensation: $" + employee.calculateCompensation());
			%				if (employee instanceof Promotable) {
				%					((Promotable) employee).promote();
				%				}
			%			}
		%		}
	%	}
%\end{lstlisting}
%
%\textbf{Pertanyaan Refleksi:}
%\begin{itemize}
%	\item Bagaimana polimorfisme memungkinkan kita untuk menangani berbagai tipe karyawan melalui referensi \texttt{Employee} tunggal?
%	\item Bagaimana pendekatan ini mendukung skalabilitas jika tipe karyawan baru diperkenalkan?
%	\item Apa keuntungan polimorfisme dalam sistem yang mengelola berbagai jenis entitas seperti karyawan?
%\end{itemize}
